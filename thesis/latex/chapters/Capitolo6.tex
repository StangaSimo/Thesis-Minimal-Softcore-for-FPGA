\chapter{Conclusioni}

L' obbiettivo principale di questa tesi è stato lo sviluppo di un semplice interprete in grado di eseguire un sottoinsieme delle istruzioni del processore Microblaze, compilarlo ed eseguirlo sulla FPGA.  Sono state create differenti versioni del interprete per dimostrare la flessibilità e configurabilità di questa soluzione. Inoltre una delle versioni dell'interprete è stata usata per simulare una GPU con core indipendenti e privi di controllore SIMD.
Il funzionamento di ciascuna versione dell'interprete è stato verificato e validato attraverso test ad-hoc specifici.
Inoltre è stata mostrata e discussa l'occupazione di ogni variante dell'interprete sulla FPGA. Questo ha permesso di stimare il numero massimo di istanze che è possibile inserire all'interno del dispositivo.
In conclusione tutti gli obbiettivi sono stati raggiunti con successo.

\section{Bilancio Personale}
Il lavoro svolto per questa tesi è stato impegnativo e stimolante. L'Affrontare un contesto nuovo come quello delle FPGA, senza particolari conoscenze in materia, ha richiesto un piccolo sforzo nell'acquisire dimestichezza con gli strumenti e le metodologie usate, detto questo la documentazione disponibile grazie alla sua chiarezza e completezza ha costituito un grandissimo supporto durante questo processo.  Inoltre, non è mancata la sfida nel risolvere i problemi che si sono presentati lungo il percorso. Tuttavia, ho potuto sempre contare sul supporto del mio professore. La sua disponibilità e competenza hanno contribuito in maniera significativa al successo di questo progetto. In definitiva, sebbene il lavoro svolto abbia richiesto impegno e e dedizione, l'incontro con il mondo delle FPGA si è rivelato non solo una sfida da affrontare, ma anche un'opportunità unica di crescita personale e accademica. 

 
