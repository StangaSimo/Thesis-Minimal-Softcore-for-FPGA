% Tipo di documento. L'uso di twoside implica che i capitoli inizino sempre con la prima pagina a sinistra, eventualmente lasciando una pagina vuota nel capitolo precedente. Se questa cosa è fastidiosa, è possibile rimuoverlo. 
\documentclass[a4paper, twoside,openright]{report}

% Dimensione dei margini
\usepackage[a4paper,top=3cm,bottom=3cm,left=3cm,right=3cm]{geometry} 
% Dimensione del font
\usepackage[fontsize=13pt]{scrextend}
% Lingua del testo
\usepackage[english,italian]{babel}
% Lingua per la bibliografia
\usepackage[fixlanguage]{babelbib}
% Codifica del testo
\usepackage[utf8]{inputenc} 
% Encoding del testo
\usepackage[T1]{fontenc}
% Permette di generare testo fittizio. Mi è stato utile 
% per capire quale sarebbe stata l'impostazione del 
% testo nella pagina prima che scrivessi un determinato paragrafo
\usepackage{lipsum}
% Per ruotare le immagini
\usepackage{rotating}
% Per modificare l'header delle pagine 
\usepackage{fancyhdr}               

% Librerie matematiche
\usepackage{amssymb}
\usepackage{amsmath}
\usepackage{amsthm}         
\usepackage{multicol}         

% Uso delle immagini
\usepackage{graphicx}
% Uso dei colori
\usepackage[dvipsnames]{xcolor}         
% Uso dei listing per il codice
\usepackage{listings}          
% Per inserire gli hyperlinks tra i vari elementi del testo 
\usepackage{hyperref}     
% Diversi tipi di sottolineature
\usepackage[normalem]{ulem}

% -----------------------------------------------------------------

% Modifica lo stile dell'header
\pagestyle{fancy}
\fancyhf{}
\lhead{\rightmark}
\rhead{\textbf{\thepage}}
\fancyfoot{}
\setlength{\headheight}{12.5pt}

% Rimuove il numero di pagina all'inizio dei capitoli
\fancypagestyle{plain}{
  \fancyfoot{}
  \fancyhead{}
  \renewcommand{\headrulewidth}{0pt}
}

\lstdefinelanguage{Ass}
{
  alsoletter={.}, % allow dots in keywords
  alsodigit={0x}, % hex numbers are numbers too!
  morekeywords=[1]{ % instructions
    lb, lh, lw, lbu, lhu,
    sb, sh, sw,
    sll, slli, srl, srli, sra, srai,
    add, addi, sub, lui, auipc,
    xor, xori, or, ori, and, andi,
    slt, slti, sltu, sltiu,
    beq, bne, blt, bge, bltu, bgeu,
    j, jr, jal, jalr, ret,
    scall, break, nop
  },
  morekeywords=[2]{ % sections of our code and other directives
    .align, .ascii, .asciiz, .byte, .data, .double, .extern,
    .float, .globl, .half, .kdata, .ktext, .set, .space, .text, .word
  },
  morekeywords=[3]{ % registers
    zero, ra, sp, gp, tp, s0, fp,
    t0, t1, t2, t3, t4, t5, t6,
    s1, s2, s3, s4, s5, s6, s7, s8, s9, s10, s11,
    a0, a1, a2, a3, a4, a5, a6, a7,
    ft0, ft1, ft2, ft3, ft4, ft5, ft6, ft7,
    fs0, fs1, fs2, fs3, fs4, fs5, fs6, fs7, fs8, fs9, fs10, fs11,
    fa0, fa1, fa2, fa3, fa4, fa5, fa6, fa7
  },
  morecomment=[l]{;},   % mark ; as line comment start
  morecomment=[l]{\#},  % as well as # (even though it is unconventional)
  morestring=[b]",      % mark " as string start/end
  morestring=[b]'       % also mark ' as string start/end
}

% Stile del codice
\lstdefinestyle{codeStyle}{
    % Colore dei commenti
    commentstyle=\color{teal},
    % Colore delle keyword
    keywordstyle=\color{Magenta},
    % Stile dei numeri di riga
    numberstyle=\tiny\color{gray},
    % Colore delle stringhe
    stringstyle=\color{violet},
    % Dimensione e stile del testo
    basicstyle=\ttfamily\footnotesize,
    % newline solo ai whitespaces
    breakatwhitespace=false,     
    % newline si/no
    breaklines=true,                 
    % Posizione della caption, top/bottom 
    captionpos=b,                    
    % Mantiene gli spazi nel codice, utile per l'indentazione
    keepspaces=true,                 
    % Dove visualizzare i numeri di linea
    numbers=left,                    
    % Distanza tra i numeri di linea
    numbersep=5pt,                  
    % Mostra gli spazi bianchi o meno
    showspaces=false,                
    % Mostra gli spazi bianchi nelle stringhe
    showstringspaces=false,
    % Mostra i tab
    showtabs=false,
    % Dimensione dei tab
    tabsize=2
} \lstset{style=codeStyle}

% Stile di codice per dimensioni maggiori, in cui ho avuto bisogno di un testo più picolo (ad esempio se si vuole inserire del codice che ha linee molto lunghe). Per usare questo stile piuttosto che il precedente, usare 

% \lstset{style=longBlock}
%  % inserire il codice...
% \lstset{style=codeStyle}

% Il secondo comando consente di tornare allo stile precedente 
\lstdefinestyle{longBlock}{
    commentstyle=\color{teal},
    keywordstyle=\color{Magenta},
    numberstyle=\tiny\color{gray},
    stringstyle=\color{violet},
    basicstyle=\ttfamily\scriptsize,
    breakatwhitespace=false,         
    breaklines=true,                 
    captionpos=b,                    
    keepspaces=true,                 
    numbers=left,                    
    numbersep=5pt,                  
    showspaces=false,                
    showstringspaces=false,
    showtabs=false,                  
    tabsize=2
} \lstset{style=codeStyle}

% Togliendo il commento al comando che segue, si inseriscono nella bibliografia anche le fonti presenti in Bibliography.bib ma non citati direttamente con il comando \cite
% \nocite{*}

% Margini prima e dopo blocchi di codice, per avere più distanza
\lstset{aboveskip=20pt,belowskip=20pt}

% Modifica dello stile dei riferimenti, con il testo in cyano
\hypersetup{
    colorlinks,
    linkcolor=CornflowerBlue,
    citecolor=CornflowerBlue
}

% Aggiunti definizioni, teoremi, linea e listing
\newtheorem{definition}{Definizione}[section]
\newtheorem{theorem}{Teorema}[section]
\providecommand*\definitionautorefname{Definizione}
\providecommand*\theoremautorefname{Teorema}
\providecommand*{\listingautorefname}{Listing}
\providecommand*\lstnumberautorefname{Linea}

\raggedbottom

%\newcommand{\cgs}[1]{{\textcolor{brown}[\textcolor{red}{\bf{GS: }}{ \textcolor{brown}{#1]}}}}             
%\newcommand{\cmc}[1]{{\textcolor{blue}[\textcolor{magenta}{\bf{MC: }}{ \textcolor{blue}{#1]}}}}



% -----------------------------------------------------------------
\begin{document}

\begin{titlepage}
\begin{figure}[!htb]
    \centering
    \includegraphics[keepaspectratio=true,scale=0.5]{images/Frontespizio/cherubinFrontespizio.eps}
\end{figure}

\begin{center}
    \LARGE{UNIVERSITÀ DI PISA}
    \vspace{5mm}
    \\ \large{DIPARTIMENTO DI INFORMATICA}
    \vspace{5mm}
    \\ \LARGE{Corso di Laurea Triennale in Informatica}
\end{center}

\vspace{15mm}
\begin{center}
    {\LARGE{\bf Softcore Minimale per FPGA }}
 
\end{center}
\vspace{30mm}

\begin{minipage}[t]{0.47\textwidth}
	{\large{Relatore:}{\normalsize\vspace{3mm}
	\bf\\ \large{Prof: Marco Danelutto} }}
\end{minipage}
\hfill
\begin{minipage}[t]{0.47\textwidth}\raggedleft
	{\large{Candidato:}{\normalsize\vspace{3mm} \bf\\ \large{Simone Stanganini}}}
\end{minipage}

\vspace{30mm}
\hrulefill
\\\centering{\large{ANNO ACCADEMICO 2022/2023}}

\end{titlepage}
\include{chapters/Abstract}

\tableofcontents

% Rimuovere se non si vuole la tabella delle figure
%\listoffigures

\chapter{Introduzione}

\section{Contesto}
Nel mondo dell'informatica l'utilizzo dei dispositivi FPGA (Field-Programmable Gate Array) è diventato sempre più comune per l'accelerazione di algoritmi e applicazioni specifiche attraverso hardware "customizzato", in modo da migliorare prestazioni, flessibilità e personalizzazione. Qui entrano in gioco i processori softcore, che sono diventati una risorsa essenziale per sfruttare al massimo le potenzialità dei dispositivi FPGA. 

I processori softcore sono stati sviluppati per essere configurabili e programmabili in modo da adattarsi ad ogni specifica, questa loro caratteristica li rende adatti a una vasta gamma di compiti, dall'elaborazione dati all'elaborazione dei segnali, dall'elaborazione delle immagini all'automazione industriale, rendendo più semplice lo sviluppo di queste applicazioni anche per sviluppatori senza specifiche esperienze "hardware".

\vspace{1.3cm}

\section{Obiettivo}
Per questa mia tesi, mi è stato assegnato il compito di sviluppare un semplice interprete che sia in grado di eseguire un sottoinsieme delle istruzioni del processore Microblaze, softcore sviluppato da Xilinx e successivamente sperimentare questo softcore molto piccolo sulla FPGA. Lo scopo era quello di enfatizzare quanto questa soluzione sia flessibile e configurabile, misurarne l'occupazione sul dispositivo, verificarne il corretto funzionamento e determinare il numero massimo di istanze che possono essere inserite nel dispositivo.

In prospettiva, il softcore sviluppato dovrebbe essere utilizzato per simulare una GPU con core indipendenti e privi di controllore SIMD, mettendo in evidenza il potenziale di accelerazione offerto da questa configurazione, in particolare nella computazione di applicazioni data parallel con thread di controllo "divergenti", ovvero che nello stesso istante eseguono computazioni localmente differenti tra di loro. 


\section{Scaletta Tesi}
Il resto di questa tesi è organizzata quanto segue: 
\begin{itemize}
    \item \textbf{Capitolo 2}: Descrive gli strumenti usati per lo sviluppo del progetto.
    \item \textbf{Capitolo 3}: Fornisce una panoramica generale sul funzionamento del progetto, e riporta le possibilità di configurazione e estensione del progetto.
    \item \textbf{Capitolo 4}: Fornisce in dettaglio la spiegazione dell'implementazione delle componenti del progetto.
    \item \textbf{Capitolo 5}: Riporta i risultati ottenuti sull'occupazione della FPGA, come è stato verificato il funzionamento delle diverse versioni dell'interprete.
    \item \textbf{Capitolo 6}: Conclude la tesi riassumendo i risultati ottenuti durante il percorso di sviluppo del progetto. Inoltre offre un bilancio personale sul l'intero lavoro svolto.
\end{itemize}
\chapter{Strumenti}
In questo capitolo si descrivono gli strumenti usati per lo sviluppo del progetto: 
\begin{itemize}
    \item \textbf{Processore Microblaze}, il processore MicroBlaze è stato utilizzato come base per l'implementazione dell'interprete e del suo relativo set di istruzioni (sezione \ref{Microblaze}).
    \item \textbf{Scheda Alveo U50}, un'FPGA di Xilinx, è stata utilizzata per condurre gli esperimenti di implementazione  (sezione \ref{Alveo}).
    \item  \textbf{Software Vitis}, fondamentale per programmare e sfruttare al massimo le capacità della scheda U50  (sezione \ref{Vitis}).
    \item \textbf{Il framework OpenCL} è stato impiegato per consentire la comunicazione tra l'host e l'FPGA, sfruttando al meglio le capacità di accelerazione hardware (sezione \ref{Opencl}).
    \item \textbf{Linguaggio di programmazione C++/C}, il linguaggio di programmazione C++ è stato utilizzato per gestire le chiamate OpenCL e altre parti dell'applicazione. Inoltre, il linguaggio C è stato impiegato per scrivere l'interprete stesso . (sezione \ref{C++/C})
\end{itemize}

\clearpage 

\section{Microblaze}
%\begin{figure}[h!]
%    \centering
%    \includegraphics[scale=0.4]{images/Capitolo2/1_im.png} 
%    \caption{Logo}
%    \label{microb}
%\end{figure}

\label{Microblaze}
Il MicroBlaze è un processore a microcontrollore configurabile (soft core) progettato da Xilinx, un'azienda specializzata in dispositivi programmabili come FPGA (Field-Programmable Gate Arrays) e SoC (System on Chip).

E' stato sviluppato per essere implementato all'interno delle schede FPGA Xilinx e svolge il ruolo di un processore personalizzabile, ovvero che può essere configurato in base alle specifiche esigenze dell'applicazione.

\vspace{0,5cm}

Alcune caratteristiche possono essere riassunte come segue:  

\begin{enumerate}
    \item \textbf{Configurabilità/Flessibilità.} Gli sviluppatori possono scegliere tra le varie versioni del Softcore e selezionare le funzionalità richieste per l'applicazione specifica che cercano di sviluppare, come la cache, le varie interfacce per le periferiche, l'unita di gestione degli interrupt e altro. Questo rende il processore utilizzabile in una varietà di applicazioni, tra cui controllo industriale, comunicazioni, video, sistemi embedded, e molto altro.
    \item \textbf{Set di istruzioni.} Il MicroBlaze utilizza un set di istruzioni RISC (Reduced Instruction Set Computer) e può essere personalizzato per includere istruzioni personalizzate o estensioni per ottimizzare l'elaborazione specifica dell'applicazione.
    \item \textbf{Consumo energetico.} Il MicroBlaze è stato progettato per essere efficiente dal punto di vista energetico, il che lo rende adatto per dispositivi a batteria e sistemi embedded in cui il consumo energetico è un aspetto critico.
    \item \textbf{Supporto software.} Xilinx fornisce un ambiente di sviluppo software, come Xilinx SDK (Software Development Kit), che semplifica la programmazione e il debug di applicazioni.
\end{enumerate}

\vspace{0,5cm}

In sostanza, il MicroBlaze è un processore personalizzabile che può essere adattato per adempiere a una vasta gamma di esigenze, ed è reso più facile da utilizzare grazie al supporto software fornito da Xilinx.

All'interno di questa tesi, abbiamo utilizzato solo una piccola parte delle istruzioni di questo processore. Abbiamo scelto di non considerare le sue caratteristiche hardware in quanto non erano necessarie per dimostrare la fattibilità di questa implementazione su un dispositivo FPGA.

\clearpage 

\section{Alveo U50}
\label{Alveo}

\vspace{0,3cm}


La Alveo U50 è una scheda di accelerazione FPGA sviluppata da Xilinx, è stata creata per consentire l'implementazione di software personalizzato che può essere accelerato tramite hardware. Questa combinazione offre efficienza e velocità che non sono raggiungibili attraverso il normale metodo di programmazione della CPU che ospita l'acceleratore. 

\begin{figure}[h!]
    \centering
    \includegraphics[scale=0.4]{images/Capitolo2/2_im.png} 
    \caption{Alveo U50}
    \label{U50}
\end{figure}

In altre parole, la Alveo U50 fornisce la capacità di adattare e ottimizzare il software per ottenere prestazioni eccezionali grazie all'elaborazione hardware su FPGA.

La Alveo U50 è caratterizzata da una buona potenza di calcolo e un'architettura molto configurabile, il che la rende adatta per una varietà di settori, tra cui intelligenza artificiale, elaborazione di dati, e molto altro. Inoltre, questa scheda è progettata per ridurre al minimo il consumo energetico, il che la rende ideale per applicazioni che richiedono una gestione efficiente delle risorse.

\vspace{0,3cm}

\begin{figure}[h!]
    \centering
    \includegraphics[scale=0.3]{images/Capitolo2/3_im.png}
    \caption{Specifiche Alveo U50 \cite{sitoAlveoU50}}
    \label{Specifiche-U50}
\end{figure}

Nella Tabella \ref{Specifiche-U50} si trovano riassunte le principali caratteristiche e "dimensioni della scheda.
Abbiamo scelto questa scheda perché era quella a disposizione nella macchina host usata per gli esperimenti di implementazione.
\vspace{0,3cm}

\section{Vitis}
\label{Vitis}
Come scritto nella documentazione:

\vspace{0,2cm}

"The AMD Vitis™ software platform is a development environment for developing designs that includes FPGA fabric, Arm® processor subsystems, and AI Engines" \cite{sitoAMDvitis}. 

\vspace{0.4cm}

\noindent In particolare Vitis Software Platform mette a disposizione i seguenti tool:

\begin{enumerate}
    \item \textbf{Vitis Embedded} – per scrivere applicazioni in C/C++ e farle eseguire in codice per processori arm su piattaforme embedded.
    \item \textbf{Compiler and simulators} – per implementare applicazioni usando l'AI Engine array.
    \item \textbf{Vitis HLS} – per scrivere applicazioni C/C++ basate su gli IP blocks che hanno come target di esecuzione le schede FPGA.
    \item \textbf{Vitis Model Composer} – "A model-based design tool that enables rapid design exploration within the MathWorks Simulink® environment"\cite{sitoAMDvitis}.
    \item Un set di funzioni opensource tipo  DSP, Vision, Solver, Ultrasound, BLAS, e altre ancora, che possono essere utilizzate nelle FPGA o con gli AI Engines in applicazioni "custom" .
\end{enumerate}

\vspace{0.3cm}
\noindent Contrariamente all'approccio tradizionale RTL (Register-Transfer Level) di progettazione hardware, Vitis utilizza HLS (High-Level Synthesis), che rappresenta un cambiamento radicale di paradigma  nella progettazione di FPGA.

\vspace{0.3cm}

\noindent Questo perché l'approccio HLS permette di scrivere applicazioni utilizzando codice in linguaggi di alto livello come C, C++, o OpenCL, invece di descrivere esplicitamente la logica hardware, cosi da tradurre il codice in un circuito hardware eseguibile dalla FPGA senza una vera e propria programmazione dettagliata di RTL.

\vspace{0.3cm}

\noindent Questo approccio rende accessibile la potenza di elaborazioni di un FPGA a sviluppatori che non hanno una profonda conoscenza del linguaggio RTL, cosi anche da rendere più semplice lo sviluppo di applicazioni in settori come l'intelligenza artificiale, elaborazione di immagini, e altro ancora.

\clearpage 

\section{OpenCL}
\label{Opencl}

"OpenCL™ (Open Computing Language) is an open, royalty-free standard for cross-platform, parallel programming of diverse accelerators found in supercomputers, cloud servers, personal computers, mobile devices and embedded platforms." \cite{sitoOpencl}

\begin{figure}[h!]
    \centering
    \includegraphics[scale=0.2]{images/Capitolo2/4_im.png}
    \caption{OpenCL logo}
    \label{Specifiche-U50}
\end{figure}

Il framework messo a disposizione da OpenCL crea uno standard per lo sviluppo di applicazioni parallele che richiedono di sfruttare a pieno la potenza dei calcolatori presenti al giorno di oggi, i quali sono molto eterogenei.

\vspace{0,3cm}

OpenCL velocizza l'esecuzione delle applicazioni eseguendo il codice più dispendioso dal punto di vista computazionale in acceleratori, in questa maniera gli sviluppatori possono scrivere dei kernel in C/C++, che saranno caricati tramite un device compiler per l'esecuzione parallela sui dispositivi di accelerazione.

\vspace{0,3cm}
\vspace{0,3cm}

\noindent Un applicazione OpenCL è divisa in due parti: 
\begin{itemize}
    \item \textbf{host:} questa parte è scritta in un un linguaggio come C o C++, e compilata con i compilatori tradizionali per essere eseguita sulla CPU del calcolatore host.
    \item \textbf{device:} è la parte che può essere compilata "on the fly" (ovvero tramite l'utilizzo di chiamate speciali dell'API a tempo di esecuzione), oppure si può compilare prima che l'applicazione vada in esecuzione cosi da rendere portabile il binario generato in una rappresentazione intermedia chiamata SPIR-V di Khronos. 
\end{itemize}

\begin{figure}[h!]
    \centering
    \includegraphics[scale=0.3]{images/Capitolo2/5_im.png}
    \caption{Traditional vs OpenCL programming paradigm}
    \label{funzionamentoOpenCL}
\end{figure}

\clearpage

\section{C++/C}
\label{C++/C}
I linguaggi di programmazione C e C++ sono due dei linguaggi più influenti e più utilizzati nella storia dell'informatica. Sono noti per la loro potenza, versatilità e velocità e sono utilizzati in una grande varietà di applicazioni, dai sistemi operativi al mondo del software per applicazioni generali. 

\vspace{0,3cm}
\vspace{0,3cm}

\noindent Di seguito un'introduzione a entrambi:

\vspace{0,3cm}
\noindent \textbf{Linguaggio C}:
\begin{itemize}
    \item È stato sviluppato negli anni 70 da Dennis Ritchie, ed è stato uno dei primi linguaggi ad alto livello della storia dell'informatica.
    \item È noto per la sua semplicità ed efficienza nell'accesso diretto alla memoria, ed è apprezzato per la possibilità di scrivere codice altamente ottimizzato.
    \item È ampiamente utilizzato per lo sviluppo di applicazioni embedded, sistemi operativi, compilatori e molte altre applicazioni a basso livello.
\end{itemize}

\vspace{0,3cm}
\noindent \textbf{Linguaggio C++}:
\begin{itemize}
    \item C++ è una versione estesa di C sviluppata negli anni '80 da Bjarne Stroustrup, tra le varie feature che aggiunge, c'è il concetto di programmazione orientata agli oggetti (OOP), il quale consente la creazione di software più strutturato e modulare rispetto al linguaggio C.
    \item È ampiamente utilizzato nell'industria del software, nei videogiochi, nella elaborazione di immagini e molto altro.
    \item Dispone di una libreria standard che presenta molti strumenti per lo sviluppo software. 

\end{itemize}

\vspace{0,3cm}
\vspace{0,3cm}

\noindent Entrambi sono linguaggi popolari, ma la scelta tra i due spesso dipende sia dall'applicazione specifica che dalle preferenze personali.
Tutti e due offrono un grado alto di controllo sulla macchina, ma C++ aggiunge il paradigma ad oggetti e una sintassi diversa per affrontare problemi complessi in modo più efficiente.

\vspace{0,3cm}

\noindent Abbiamo scelto di usare i linguaggi C e C++ per due motivi, in primo luogo perché rientrano tra la lista di linguaggi supportati dall'ambiente Vitis e dal framework OpenCL, inoltre la mia familiarità con questi ha facilitato lo sviluppo del progetto.





\chapter{Progetto Logico}
In questo capitolo viene fornita una panoramica generale sul funzionamento del progetto e del comportamento delle sue parti principali. Inoltre viene data un introduzione sull'estensibilità e la personalizzazione sia dell'interprete che della configurazione della scheda FPGA. 

\vspace{0.2cm}

\begin{figure}[h!]
    \centering
    \includegraphics[scale=0.6]{images/Capitolo3/5_im.png}
    \caption{Schema Generale}
    \label{Graficogenerale}
\end{figure}

\clearpage

Come si può osservare dal grafico \ref{Graficogenerale}, il progetto si basa sull'utilizzo un interprete scritto in un linguaggio di alto livello. Questo interprete attraverso la toolchain di Vitis viene trasformato in un kernel eseguibile all'interno dell'FPGA, di cui possono essere istanziate diverse copie (Control Unit), al fine di eseguire programmi anche diversi tra loro.

\vspace{1cm}

Insieme alla configurazione dell'interprete è necessario fornire un file scritto nel linguaggio assembly usato dall'interprete, il quale contiene le istruzioni che saranno eseguite dall'acceleratore, ovvero il programma da eseguire sul softcore. Durante l'esecuzione del Kernel, queste istruzioni presenti nel file assembler vengono caricate nella memoria allocata per il softcore sulla FPGA.

Dopo aver eseguito i calcoli sulla FPGA si otterranno i risultati desiderati. Questi risultati saranno estratti dalla memoria del softcore nella FPGA, e successivamente si potranno osservare nella memoria della macchina host.

\vspace{1cm}

Come si può notare dal grafico, questo progetto offre una grande possibilità di configurazione, con la capacità di effettuare modifiche nelle seguenti aree: 

\begin{itemize}
    \item \textbf{Interprete}: essendo un interprete scritto in linguaggio di alto livello, offre un alto grado di personalizzazione. È possibile specializzarlo scegliendo solo il set di istruzioni più adatto per l'applicazione specifica, oppure estenderlo per interpretare un insieme più grande di istruzioni assembly, o persino cambiare totalmente l'architettura interpretata.
    \item \textbf{Control Unit}: durante il processo della compilazione del kernel, è possibile scegliere quante istanze (CU) allocare all'interno della FPGA. Questo, permette di avere più "core logici" all'interno della scheda FPGA, da gestire in base alle esigenze dell'applicazione.
    \item \textbf{Assembly.s}: questo file, a patto che rispetti il set di istruzioni dell'interprete sviluppato, offre un grande libertà nella scrittura del codice, può contenere qualsiasi flusso di istruzioni consentito dall'interprete. Inoltre è possibile usare file assembly diversi per ciascuna delle CU, cosi da aver più core che eseguono flussi di istruzioni diverse. 
\end{itemize}

\vspace{1cm}

Inoltre (come vedremo nel capitolo Implementazione, Cap. \ref{implementazione}), l'interprete funziona con una memoria dati, e una memoria per i registri, le quali sono altamente anch'esse configurabili in modo da poter soddisfare esigenze diverse.

\clearpage

\section{Funzionamento}
\label{funzionamento}

\noindent Di seguito un grafico sul funzionamento generale delle componenti del progetto.

\begin{figure}[h!]
    \centering
    \includegraphics[scale=0.5]{images/Capitolo3/6_im.png}
    \caption{Grafico Funzionamento}
    \label{funzionamentoOpenCL}
\end{figure}

Nel lato hardware del progetto, l'attività si svolge su una macchina host dove in cui è installata la scheda FPGA (la FPGA Board a sinistra in fig. \ref{funzionamentoOpenCL}). Le componenti software del progetto (a sinistra in fig. \ref{funzionamentoOpenCL}) sono suddivise principalmente in tre parti:

\begin{itemize}
    \item \textbf{FPGA Binary}, questo file contiene il Bit Stream FPGA, il quale è l'implementazione hardware del kernel definito nel file \texttt{interprete.cpp}, questo kernel verrà eseguito successivamente dal chip della scheda FPGA.
    \item \textbf{Microblaze Bytecode}, il file contenente il bytecode generato dal compilatore \texttt{mb-gcc}, utilizzando le istruzioni scritte in linguaggio assembly Microblaze, il quale sarà interpretato dal kernel presente nella FPGA.
    \item \textbf{Host Executable}, questo eseguibile compilato tramite \texttt{g++}, viene eseguito sulla macchina host, ed è responsabile di effettuare le chiamate API di OpenCL, istanziare e caricare i dati nella memoria dell'FPGA (DDR), e gestire i risultati ottenuti dall'interpretazione delle istruzioni assembly.
\end{itemize}

\begin{figure}[h!]
    \centering
    \includegraphics[scale=0.5]{images/Capitolo3/2_im.png}
    \caption{Flusso di Compilazione Interprete}
    \label{funzionamentoOpenCL}
\end{figure}

\noindent Per la generazione del Bit Stream, partiamo da un file chiamato \textbf{interprete.cpp}, dove è definita la funzione principale che verrà eseguita nella scheda FPGA.

All'interno di questo kernel è implementato l'intero interprete del processore Microblaze, il quale legge i dati dalla memoria globale della FPGA, ovvero la memoria dati, la memoria delle istruzioni da interpretare (precedentemente caricate dalla parte host), e la memoria dei registri.
L'Interprete successivamente esegue le istruzioni e trasferisce la sua memoria dati nuovamente nella memoria globale della FPGA.

La creazione di questo file inizia da un file con all'interno una funzione \texttt{extern C}, il quale viene compilato tramite il compilatore fornito dalla toolchain di Vitis chiamato \texttt{v++}, con l'aggiunta della flag \texttt{-c}, la quale specifica di per compilare il codice sorgente in un file \texttt{.xo}, il quale contiene tutti i dati necessari per la successiva generazione del bitstream. Notare che questo processo, che traduce il codice di alto livello del kernel nel linguaggio RTL, richiede un tempo dell'ordine dei minuti o anche secondi, quindi relativamente breve.

Successivamente si compila usando nuovamente \texttt{v++} utilizzando la flag \texttt{-l}, cosi da effettuare il linking del kernel compilato dal file \texttt{.xo}, con la piattaforma target, in modo da generare il bitstream che è contenuto nel file \texttt{.xclbin}, che contiene tutti i dati necessari per configurare l'FPGA con un kernel che implementa l'interprete. È da notare che la durata di questa fase può variare da minuti a ore, in base alla complessità e alla quantità di codice coinvolto. La durata di questo passaggio può variare da minuti a ore di tempo, a seconda della complessità e quantità di codice.


\clearpage

\begin{figure}[h!]
    \centering
    \includegraphics[scale=0.6]{images/Capitolo3/3_im.png}
    \caption{Flusso Assembler}
    \label{flussoassembler}
\end{figure}

\noindent Per la creazione del bytecode si inizia partendo da un file scritto in linguaggio assembly (vedi fig. \ref{flussoassembler}). 

Successivamente usando il compilatore fornito dalla toolchain di Xilinx chiamato \texttt{mb-gcc}, questo file viene compilato in un file "oggetto" \texttt{.o} . Questo file contiene il risultato della compilazione insieme ad altri meta dati aggiunti dal compilatore stesso.

Successivamente tramite l'utilizzo del tool chiamato \texttt{obj-dump} estraiamo la parte \texttt{.text} dal file. Questa sezione contiene le istruzioni assembly tradotte dal compilatore e convertite in uno stream di byte.

\begin{figure}[h!]
    \centering
    \includegraphics[scale=0.35]{images/Capitolo3/4_im.png}
    \caption{Flusso Host}
    \label{flussohost}
\end{figure}

\noindent Un'ultima parte del progetto riguarda la compilazione dell'eseguibile che sarà eseguito sulla CPU della macchina host (vedi fig.\ref{flussohost}). Questo file svolge un ruolo cruciale nella gestione della FPGA, compresa l'inizializzazione della memoria esterna dove verranno presi i dati, che includono i registri, la memoria dati e memoria delle istruzioni. 

L'eseguibile è responsabile inoltre di caricare (tramite OpenCL) i kernel precedentemente compilati all'interno della scheda FPGA e successivamente di avviare il processo di esecuzione dell'interprete e verifica i risultati ottenuti dall'elaborazione. 

\chapter{Implementazione}
\label{implementazione}

In questo capitolo spiegheremo in dettaglio l'implementazione delle componenti del progetto. Cominceremo con un'analisi del Interprete del SoftCore e di tutte le sue componenti, per poi spiegare i cambiamenti richiesti per far funzionare il codice dentro un kernel compilabile per una FPGA. Successivamente descriveremo in dettaglio l'interfaccia lato host, e come si svolge il processo di compilazione del progetto. Esploreremo una versione dell'interprete con i registri in virgola Mobile (floating-point).
Verrà analizzato come l'interfaccia host cambia con l'istanziazione di più Control Unit.
Si conclude con una analisi della versione dell'interprete progettata per emulare una GPU senza controllore SIMD.

\clearpage

\section{Interprete Softcore}
\label{Interprete Softcore}
In questa sezione, esploreremo le componenti dell'interprete.
È necessario sottolineare che questa è la versione progettata per la compilazione ed esecuzione su una qualsiasi macchina host e scritta interamente nel linguaggio C, le modifiche necessarie per l'esecuzione sulla FPGA saranno dettagliate nella prossima sezione \ref{Interprete Kernel}.

\vspace{0.5cm}

\noindent Il codice completo relativo alla sezione seguente è presente nell'appendice (ref a cpu.c e cpu.h)

\vspace{0.5cm}

Per lo sviluppo di questo interprete è stata seguita la documentazione ufficiale del Softcore Microblaze \cite{sitoMicroblaze}.

\vspace{0.5cm}

Nonostante questo interprete non replichi il funzionamento hardware effettivo del processore Microblaze, comunque rispetta il fondamentale paradigma di qualsiasi processore. Questo si basa sull'utilizzo di registri per mantenere lo stato della CPU e portare avanti la computazione e su l'uso di una memoria per immagazzinare e recuperare i dati.

È importante evidenziare che nella implementazione seguente si è optato per l'utilizzo di bit field per le strutture dati, cosi da migliorare la chiarezza e leggibilità delle dimensioni dei singoli campi, mantenendo al fedeltà alla documentazione. 

\vspace{0.6cm}

\noindent \textbf{Registri:}

\vspace{0.2cm}

\noindent Cominciamo illustrando la struttura dei registri, partendo dalla loro implementazione:
\begin{lstlisting}[language=C]
struct Registers
{
    int32_t r[32];
    int16_t im;
    bool c : 1;
    int32_t pc;
};
\end{lstlisting}
\label{structreg}

\noindent Come è possibile vedere dalla implementazione, il processore dell'interprete è dotato dei seguenti registri: 
\begin{itemize}
    \item \texttt{r}: un array da $32$ registri interi a \texttt{32 bit}.
    \item \texttt{im}: ovvero un registro da \texttt{16 bit} il quale viene usato dalla istruzione \texttt{imm} (\ref{imm}) per estendere l'immediato precedentemente istanziato da \texttt{16} a \texttt{32 bit}.
    \item \texttt{c}: ovvero una flag da \texttt{1 bit} il quale segnala il caso in cui l'istruzione precedente ha generato un overflow. 
    \item \texttt{pc}: Program Counter, un registro che indica la prossima istruzione da eseguire.
\end{itemize}    

\noindent \textbf{Memoria:}

\begin{lstlisting}[language=C]
struct Memory
{
    int32_t *data;
    int32_t size;
};
\end{lstlisting}

\vspace{0.2cm}

\noindent La memoria è dotata di un array di interi a \texttt{32 bit}, con la relativa \texttt{size} per indicarne la dimensione.

\vspace{0.5cm}

\noindent \textbf{Istruzioni:}

\vspace{0.2cm}

\noindent L'interprete adotta lo standard di istruzioni RISC, le quali sono caratterizzate tutte da una lunghezza di 32 bit. Esistono due tipi di istruzioni, differenziate in base alla loro natura:

\begin{itemize}
    \item Type A:  
    \begin{figure}[h!]
    \centering
    \includegraphics[scale=0.35]{images/Capitolo4/1_im.png}
    \caption{Type A istruction}
    \label{typeA}
    \end{figure}

    \item Type B:
    \begin{figure}[h!]
    \centering
    \includegraphics[scale=0.35]{images/Capitolo4/2_im.png}
    \caption{Type B istruction}
    \label{typeB}
    \end{figure}
\end{itemize}

\noindent Dove le componenti sono:

\begin{itemize}
    \item \texttt{Opcode}:\texttt{6 bit} per distinguere il tipo di istruzione.
    \item \texttt{rd}: \texttt{5 bit} per identificare il registro nel quale memorizzare il risultato dell'istruzione.
    \item \texttt{ra} e \texttt{rb}: ciascuno da \texttt{5 bit} come nel caso di \texttt{rd}, per identificare i due registri dove nel caso di una istruzione di tipo A, vengono utilizzati per eseguire l'operazione richiesta dall'istruzione. 
    \item \texttt{imm}: costituito \textbf{16 bit}, è presente sono nelle istruzioni di tipo B e viene utilizzato come fosse il registro \texttt{rb} nel caso delle istruzioni di tipo A. Questi bit identificano un valore immediato inserito direttamente nelle istruzioni assembler. Nel caso l'istruzione sia stata preceduta da una istruzione imm (\ref{imm}), viene esteso a 32 bit.
\end{itemize}

\vspace{0.3cm}

\noindent Di seguito l'implementazione della struttura che rappresenta le istruzioni:

\begin{lstlisting}[language=C]
struct Instruction 
{
    int8_t type : 1; /* 1 type A, 0 type B */
    int8_t opcode : 6;
    int8_t rd : 5;
    int8_t ra : 5;
    int8_t rb : 5;
    int32_t im : 32;
};
\end{lstlisting}
\label{structinstr}

\vspace{0.2cm}

\noindent L'interprete acquisisce le istruzioni da un file \texttt{.text}( sez \ref{compilazione}).  Queste istruzioni vengono caricate nell'interprete tramite la funzione (\ref{} ref apice) sottoforma di un array di puntatori a 4 interi di \texttt{8 bit}. Per elaborare le istruzioni l'interprete effettua il parsing di tali array per caricarli nella struttura \ref{structinstr} tramite la seguente funzione:

\begin{lstlisting}[language=C,label={parseinstr},caption={Parse Instruction}]
struct Instruction *parse_instruction(int8_t *instr, int8_t type, struct Instruction *res, int16_t *im)
{
	if (type) /* Type A */
	{
		res->type = type;
		res->rd = (instr[0] << 3) + ((instr[1] >> 5) & 0b00000111);
		res->ra = instr[1] & 0b00011111;
		res->rb = (instr[2] >> 3) & 0b00011111;
	}
	else /* Type B */
	{
		res->type = type;
		res->rd = ((instr[0] << 3) & 0b00011000) + ((instr[1] >> 5) & 0b00000111);
		res->ra = instr[1] & 0b00011111;
		int16_t n = instr[2];
		n = (n << 8) + (((int16_t)instr[3]) & 0b0000000011111111);

		if (*im) /* imm istruction before */
		{
			res->im = (*im << 16) + ((int32_t)n & 0b00000000000000001111111111111111);
			*im = 0;
		}
		else
			res->im = (int32_t)n;
	}
	return res;
}
\end{lstlisting}

Questa funzione accetta come parametri l'array della singola istruzione, il tipo della istruzione determinato a priori in base all'opcode, la struttura delle istruzioni da restuire dopo il parsing, e l'intero im (ovvero il registro in fig. \ref{structreg}). Nel caso l'istruzione sia preceduta da una istruzione imm (reference a imm),  come indicato nell'implementazione, il parametro viene usato per estendere l'immediato presente nella struttura a 32 bit. Questo viene fatto utilizzando i \texttt{16 bit} meno significativi dei bit dell'istruzione corrente, e i restanti bit più significativi dalla istruzione precedente imm. 
Si può notare come nel caso ci sia una istruzione di tipo A si rispetta la forma specificata in fig. \ref{typeA}, mentre nel caso di tipo B, si usa la specifica della fig. \ref{typeB}.

\vspace{0.3cm}

Prima di spiegare come alcune delle singole istruzioni sono state implementate, l'interprete al suo interno contiene delle funzioni di supporto per facilitare la scrittura di ogni singola implementazione di istruzione.

La prima funzione è la seguente: 
\begin{lstlisting}[language=C,label={updatepc},caption={Update PC}]
void update_PC(struct Registers *reg, int32_t n, bool delay)
{
    if (!delay)
        reg->pc = reg->pc + n / 4;
}
\end{lstlisting}

Questa funzione viene utilizzata per aggiornare lo stato del registro del Program Counter. Per facilità viene trattato come un numero intero. Tuttavia nella realtà e anche per il compilatore, per avanzare di un istruzione (dato che ogni istruzione occupa 4 byte) dividiamo il valore di 4. Questo permette di mantenere coerenza tra la rappresentazione del PC register nell'interprete e i comandi del compilatore.

Inoltre la CPU dispone istruzioni di salto che possono includere un flag di Delay Slot (\cite{libroarchitetture}). In caso questo flag di \texttt{Delay Slot} sia attivo, indipendentemente dal risultato del controllo dell'istruzione di salto, viene comunque eseguita l'istruzione successiva a quella in corso, a patto che questa non modifichi il Program Counter (come specificato nella documentazione \cite{sitoMicroblaze}. Tramite l'utilizzo di questo parametro nella funzione \texttt{update\_PC} manteniamo una corretta esecuzione del flusso del codice.

\clearpage

\noindent Un'altra funzione di supporto all'esecuzione dell'interprete è la seguente:
\begin{lstlisting}[language=C,label={convreg}]
int8_t conv_reg(int8_t n)
{
    return n & 0b00011111;
}
\end{lstlisting}

Questa funzione viene usata nel caso in cui il valore presente nella \texttt{struct Instructon} \ref{structinstr}, deve essere usato per operazioni bit a bit. 

Queste operazioni potrebbero non essere eseguite correttamente senza questa specifica conversione. Ciò è dovuto al fatto che qualsiasi operatore del linguaggio C, quando applicato come nel nostro caso a un intero \texttt{int8\_t}, il quale in realtà è un bit field da \texttt{5 bit} (come si nota in fig. \ref{structinstr}), utilizza tutti gli \texttt{8 bit} estendendo il nostro campo con dei bit non appropriati da \texttt{5} a \texttt{8}. Questo può portare a risultati inaspettati durante le operazioni.

Per risolvere questo problema, la funzione \texttt{conv\_reg} esegue una conversione, impostando correttamente i bit mancanti nel nostro campo. Questo assicura un utilizzo corretto degli operatori del linguaggio C senza gli errori derivanti da un estensione errata dei bit.

\vspace{0.3cm}

\noindent Come si può leggere nella documentazione \cite{sitoMicroblaze}, l'operazione fondamentale della CPU consiste nella somma di due registri. Questa operazione costituisce la base per gran parte delle istruzioni ed è l'unica che può causare overflow. Per gestire questo aspetto è stata implementata la seguente funzione:
\begin{lstlisting}[language=C]
int32_t add_Check_Overflow(int32_t a, int32_t b, bool *c)
{
    int64_t res = (int64_t)a + (int64_t)b; /* only last 32 bit*/
    if (res > INT32_MAX)
    {
        *c = true;
        res = INT32_MAX;
    }
    else if (res < INT32_MIN) /* underflow */
    {

        *c = true;
        res = INT32_MIN;
    }
    else
    {
        *c = false;
    }

    return (int32_t)res;
}
\end{lstlisting}
Questa funzione accetta due interi a \texttt{32 bit}, e il booleano \texttt{c}, il quale rappresenta il flag di carry dei registri, che viene settato in base a se si è o no verificato l'Overflow. 
Questi due interi vengono convertiti a \texttt{64 bit} per effettuare l'addizione, per poi controllare il risultato per vedere se sfora il massimo valore degli interi da \texttt{32 bit} oppure no.
Nel caso di questo interprete viene anche gestito l'Underflow.

La base dell'interprete stesso è una funzione chiamata \texttt{run\_instruction}. All'interno di questa funzione è presente uno switch, il quale in base all'opcode situato nei primi \texttt{6 bit} dell'istruzione, seleziona il caso appropriato per eseguire l'operazione corrispondente. 

\vspace{0.3cm}
\noindent L'implementazione della funzione è la seguente:

\begin{lstlisting}[language=C,caption={Run Instruction},label={runinstruction}]
void run_instruction(int8_t *instruction, 
                    struct Memory *data, 
                    struct Registers *reg, 
                    int8_t **instructions, 
                    bool delay)
{
	struct Instruction *instr = malloc(sizeof(struct Instruction));
	bool carry = 0; //carry
    op_code = (instruction[0] >> 2) & 0b00111111;
	instr->opcode = op_code;
	int32_t delayed_instruction, addr;
	int8_t branch_type, is_delayed, is_absolute, is_link;

	switch (op_code) {
		...
    }
}
\end{lstlisting}
Questa funzione accetta come parametri l’array della singola istruzione, lo stato corrente della memoria e dei registri, l'array contenente tutte le istruzioni (che viene utilizzato nelle istruzioni con delay per effettuare una ricorsione, come nella fig. \ref{branch}), e un booleano Delay Slot. Quando viene eseguita un istruzione con delay, il parametro impedisce l'aggiornamento del Program Counter nella funzione \texttt{update\_pc}. (come visto nella fig. \ref{updatepc})

\vspace{0.3cm}

All'inizio della funzione viene creata un istanza che rappresenta le istruzioni dopo il parsing della funzione \texttt{parse\_instruction} (fig. \ref{parseinstr}).
All'interno di questa struttura, il parametro \texttt{op\_code}, viene inizializzato per motivi di facilità, poiché è utilizzato immediatamente. Successivamente vengono create le istanze delle variabili utilizzate all'interno dei vari case dello switch, le quali verranno spiegate successivamente.

\vspace{0.3cm}

Segue una spiegazione dei vari casi dello switch. Per evitare di ripetizioni, verranno illustrate solo le parti principali, poiché la maggior parte degli altri casi presentano solo delle variazioni minori.

\vspace{0.3cm}

La prima funzione dello switch è la \texttt{ADD} ed è implementata come segue:
\begin{lstlisting}[language=C]
  case 0x0 : 
  { /* ADD 000000 */
    instr = parse_instruction(instruction, TYPE_A, instr, &reg->im);
    reg->r[instr->rd] = add_Check_Overflow(reg->r[instr->ra], reg->r[instr->rb], &carry);
    update_PC(reg, 4, delay);
    break;
  }
\end{lstlisting}
Questo caso rappresenta in gran parte la struttura di tutte le istruzioni. Notare che i casi sono abbinati agli opcode scritti in esadecimale per una questione di leggibilità. L'esecuzione inizia chiamando la funzione per il parsing dell'istruzione, con i parametri appropriati. Secondo la documentazione, l'istruzione \texttt{ADD} è una di tipo A, per cui si chiama con la variabile \texttt{TYPE\_A} definita all'inizio dell'interprete. Successivamente per eseguire l'operazione effettiva, viene utilizzata la funzione di supporto \texttt{ add\_Check\_Overflow}, che assicura la gestione dell'overflow e dell'eventuale bit di carry.

È importante notare che, nel caso dell'istruzione \texttt{ADD}, la CPU Microblaze non tiene conto del bit di carry né per fare la somma e né per salvarlo in caso di carry effettivo.
Possiamo osservare che il funzionamento effettivo consiste nell'eseguire la somma del contenuto del registro \texttt{ra} con il registro \texttt{rb}, e successivamente di memorizzare il risultato nel registro \texttt{rd}.
Si osserva come al termine dell'esecuzione, venga effettuato l'aggiornamento del Program Counter per puntare alla istruzione successiva.

\vspace{0.3cm}

\noindent Per gestire questa situazione, è utile osservare l'implementazione che segue dell'istruzione \texttt{ADDCK}:
\begin{lstlisting}[language=C]
  case 0x6:
  { /* ADDCK 000110 */
   instr = parse_instruction(instruction, TYPE_A, instr, &reg->im);
   reg->r[instr->rd] = add_Check_Overflow(reg->r[instr->ra], reg->r[instr->rb] + reg->c, &carry);
   reg->c = carry;
   update_PC(reg, 4, delay);
   break;
  }
\end{lstlisting}
Possiamo notare come la struttura del rimane invariata rispetto alla istruzione. Come anche il nome dell'istruzione implica, questa variante dell'istruzione \texttt{ADD}, tramite il flag \texttt{C} (nel nome della istruzione) tiene conto del bit di carry precedentemente salvato, per effettuare l'operazione di somma, e con il flag \texttt{K} ovvero "keep", implica il salvataggio dell'eventuale bit di carry nello stato dei registri generato dall'operazione.

\noindent Di seguito l'implementazione dell'istruzione \texttt{RSUB}: 
\begin{lstlisting}[language=C]
 case 0x1:
 { /* RSUB 000001 */
    instr = parse_instruction(instruction, TYPE_A, instr, &reg->im);
    reg->r[instr->rd] = add_Check_Overflow(reg->r[instr->rb], add_Check_Overflow(~reg->r[instr->ra], 1, &carry), &carry);
    update_PC(reg, 4, delay);
    break;
 }
\end{lstlisting}
Si noti una differenza fondamentale rispetto alla istruzione \texttt{ADD}: qui viene eseguita un somma tra il contenuto del registro \texttt{rb} e il not del contenuto del registro \texttt{ra} sommato a 1. Ovvero la sottrazione è implementata come somma del complemento a due del secondo operando. Questo approccio per effettuare la sottrazione è stato adottato per rispettare il funzionamento specificato nella documentazione \cite{sitoMicroblaze}.
Nonostante questo cambiamento il paradigma di esecuzione rimane invariato. Da notare che anche questa istruzione presenta le varianti \texttt{RSUBC}, \texttt{RSUBK} e \texttt{RSUBCK}.

\vspace{0.3cm}

\noindent Sia l'istruzione \texttt{ADD} che l'istruzione \texttt{RSUB} presentano le rispettive varianti \texttt{ADDI} e \texttt{RSUBI} per gestire il caso dei valori immediati, entrambe con le varianti con i vari flag per il bit di carry presenti. Di seguito, l'implementazione dell'istruzione \texttt{ADDI}:

\begin{lstlisting}[language=C]
 case 0x8:
 { /* ADDI 001000 */
    instr = parse_instruction(instruction, TYPE_B, instr, &reg->im);
    reg->r[instr->rd] = add_Check_Overflow(instr->im, reg->r[instr->ra], &carry);
    update_PC(reg, 4, delay);
    break;
 }
\end{lstlisting}
Possiamo notare pur essendo la stessa istruzione, al posto del registro \texttt{rb} è presente il valore immediato \texttt{instr->im}, il quale viene utilizzato come operando per l'operazione di somma. 
Per il resto il flusso dell'esecuzione rimane invariata.

\vspace{0.3cm}

L'interprete presenta la possibilità di eseguire operazioni bit a bit tramite le istruzioni offerte dall'architettura RISC, tra cui \texttt{AND}, \texttt{OR}, \texttt{SRA}, \texttt{XOR}, e cosi via.
Anche queste presentano lo stesso paradigma di esecuzione delle istruzione precedenti, come è possibile osservare dall'implementazione seguente: 

\begin{lstlisting}[language=C]
case 0x22:
{ /* XOR 100010 */
    instr = parse_instruction(instruction, TYPE_A, instr, &reg->im);
    reg->r[instr->rd] = reg->r[instr->ra] ^ reg->r[instr->rb];
    update_PC(reg, 4, delay);
    break;
}
\end{lstlisting}

Nel caso in cui nel linguaggio assembly venga inserito un immediato che superi i 16  bit di grandezza, ovvero la dimensione riservata ai valori immediati nelle istruzioni di tipo B (fig. \ref{typeB}), il compilatore aggiunge, dopo il processo di compilazione, l'istruzione \texttt{imm}. Questa istruzione estende l'immediato dell'istruzione che la posticipa a 32 bit, nel modo specificato nella spiegazione della funzione \texttt{parse\_instruction} (fig. \ref{parseinstr}). L'implementazione di tale istruzione è la seguente:

\begin{lstlisting}[language=C]
 case 0x2C:
 { /* IMM 101100 */
    instr = parse_instruction(instruction, TYPE_B, instr, &reg->im);
    reg->im = instr->im;
    update_PC(reg, 4, delay);
    break;
 }
\end{lstlisting}
\label{imm}

\vspace{0.3cm} 

Le uniche istruzioni utilizzate da questo interprete per gestire la memoria dati sono \texttt{LW} e \texttt{SW}, entrambe con le varianti \texttt{LWI} e \texttt{SWI} per gestire il caso degli immediati.

\noindent Segue la loro implementazione:
\begin{lstlisting}[language=C]
 case 0x32:
 { /* LW 110010 */
    instr = parse_instruction(instruction, TYPE_A, instr, &reg->im);
    addr = (uint32_t)(reg->r[instr->ra] + reg->r[instr->rb]);
    reg->r[instr->rd] = data->data[addr];
    update_PC(reg, 4, delay);
    break;
 }
 case 0x36:
 { /* SW 110110 */
    instr = parse_instruction(instruction, TYPE_A, instr, &reg->im);
    addr = (uint32_t)(reg->r[instr->ra] + reg->r[instr->rb]);
    data->data[addr] = reg->r[instr->rd];
    update_PC(reg, 4, delay);
    break;
 }
\end{lstlisting}
Osserviamo che entrambe queste istruzioni effettuano una somma tra il registro \texttt{ra} e il registro \texttt{rb} per ottenere l'indirizzo di memoria. Nel caso dell'istruzione \texttt{LW} salviamo nel registro \texttt{rd} il contenuto della memoria all'indirizzo precedentemente calcolato dentro la variabile \texttt{addr}, mentre nel caso di \texttt{SW} facciamo esattamente il contrario, ossia salviamo in memoria il contenuto del registro \texttt{rd} all'indirizzo precedentemente calcolato.
Nella versione \texttt{SWI} e \texttt{LWI} l'unica differenza (oltre al tipo di istruzione, che diventa di tipo B), è la seguente riga:
\begin{lstlisting}[language=C]
 ...
 addr = (uint32_t)(reg->r[instr->ra] + instr->im);
 ...
\end{lstlisting}
Si nota come al posto del registro \texttt{rb} venga usato semplicemente il valore immediato.

Notare come l'indirizzo \texttt{addr} viene sempre usato come un \texttt{unsigned int} a \texttt{32 bit}. Questo perché la memoria dati è un array, e l'indicizzazione inizia da \texttt{0} e cosi via in maniera sequenziale, di conseguenza l'utilizzo di questi cast assicura la corretta esecuzione del codice.

\vspace{0.3cm}

Le istruzioni di salto gestite da questo interprete si distinguono, come le altre istruzioni, in due categorie: quelle che gestiscono gli immediati, contrassegnate dal flag \texttt{I} nel nome, e quelle senza immediati, ovvero con l'utilizzo di registri normali. 

Di seguito la loro implementazione:
\begin{lstlisting}[language=C,label={branch},caption={Istruzioni Branch}]
 case 0x27:
 { /* BEQ BGE BGT BLE BLT BNE 100111 */
    instr = parse_instruction(instruction, TYPE_A, instr, &reg->im);
    delayed_instruction = reg->pc + 1;  /* prossima istruzione da eseguire in caso di delay*/
    branch_type = conv_reg(instr->rd) & 0b00001111;
    is_delayed = conv_reg(instr->rd) & 0b00010000;

  if ((branch_type == 0x0 && reg->r[instr->ra] == 0x0) || /* BEQ D0000 */
    (branch_type == 0x5 && reg->r[instr->ra] >= 0x0) || /* BGE D0101 */
    (branch_type == 0x4 && reg->r[instr->ra] >  0x0) ||	/* BGT D0100 */
    (branch_type == 0x3 && reg->r[instr->ra] <= 0x0) || /* BLE D0011 */
    (branch_type == 0x2 && reg->r[instr->ra] <  0x0) ||	/* BLT D0010 */
    (branch_type == 0x1 && reg->r[instr->ra] != 0x0))		/* BNE D0001 */
        update_PC(reg, reg->r[instr->rb], delay);
  else
        update_PC(reg, 4, delay);

  if (is_delayed == 0x10) /* delayed slot */
    run_instruction(instructions[delayed_instruction], data, reg, instructions, true);
 break;
 }
\end{lstlisting}
Si nota come l'opcode sia lo stesso per tutte le istruzioni di branch (come specificato nella documentazione \cite{sitoMicroblaze}). Il tipo specifico di ciascuna lo si ricava dai 4 bit meno significativi del registro \texttt{rd} (come specificato nella documentazione \cite{sitoMicroblaze}).
Possiamo vedere come a seconda di quale sia il tipo di branch da eseguire, specificato nella variabile \texttt{branch\_type}, andiamo ad eseguire il controllo appropriato del contenuto del registro \texttt{ra}, per poi aggiornare il valore del pc register con il valore contenuto nel registro \texttt{rb}.
Successivamente, nel caso in cui l'istruzione abbia il flag \texttt{D} nel nome, che imposta a \texttt{1} il quinto bit del registro \texttt{rd}, si procede con l'esecuzione dell'istruzione successiva a quella del branch, senza tener conto se il controllo è andato a buon fine. Questo viene fatto impostando a true il parametro del delay slot, evitando cosi di aggiornare il pc register mentre eseguiamo questa istruzione, per poi riprendere il normale flusso del programma.

\vspace{0.5cm}

\noindent Fino a questo punto abbiamo esaminato le varie parti dell'interprete. Tuttavia nel contesto di un programma tutto ciò va utilizzato seguendo il flusso di una normale CPU. Per questo per sfruttare la funzione \texttt{run\_instruction}, si utilizza un ciclo che prosegue fino a quando program counter non raggiunge l'ultima istruzione della lista.
Di seguito è riporta l'implementazione di questo ciclo:
\begin{lstlisting}[language=C,caption={Ciclo Interprete},label={ciclointerprete}]
while (reg->pc < instructions_size) { 
    run_instruction(instructions[reg->pc], 
                        data, 
                        reg, 
                        instructions, 
                        false);
}
\end{lstlisting}

Inoltre per far funzionare l'interprete, è necessario disporre del bytecode delle istruzioni assembler generato dal compilatore, (come mostrato nella figura \ref{funzionamentoOpenCL}.
Per fare ciò, il metodo usato è il seguente. 
Partendo da un file \texttt{.s}, di seguito viene riportato un esempio di codice assembler:

\begin{lstlisting}
	.text
	.align	2
	.globl	main
	.ent	main
	.type	main, @function  

main:
	addi	r2,r0,55 
	addi	r3,r0,100
	cmp     r4,r2,r3 
	addi	r5,r0,2147483640
	
	.end	main
\end{lstlisting}
Successivamente utilizzando del compilatore fornito da Xilinx, \texttt{mb-gcc}, il codice viene compilato in un file \texttt{.o}.

\vspace{0.3cm}

\noindent Possiamo vedere il codice assembly disassemblato tramite il seguente comando: 

\begin{lstlisting}
mb-objdump -d assembler.o
\end{lstlisting}
Notare si utilizza il tool \texttt{objdump} offerto da Xilin. Il risultato del comando è il seguente:

\begin{lstlisting}
00000308 <main>:
 308:	20400037 	addi	r2, r0, 55
 30c:	20600064 	addi	r3, r0, 100
 310:	14821801 	cmp	r4, r2, r3
 314:	b0007fff 	imm	32767
 318:	20a0fff8 	addi	r5, r0, -8
\end{lstlisting}
Notare come il compilatore aggiunge l'istruzione \texttt{imm} per estendere l'immediato della istruzione \texttt{addi}. 
Possiamo notare come questa sezione del file \texttt{.o} contiene il bytecode delle istruzioni tradotte dall'assembler a codice macchina.
A questo punto è possibile estrarre solamente la sezione \texttt{.text} attraverso l'utilizzo del tool \texttt{objcopy}. 

\vspace{0.3cm}

\noindent Questo processo avviene attraverso i comandi seguenti:
\begin{lstlisting}[language=Bash]
	mbgcc -o assemler.o -c assembler.s 
	objcopy -j .text -O binary -I elf32-little assembler.o assembler.text  
\end{lstlisting}
\label{estrazioneBytecode}

\section{Interprete Versione Kernel}
\label{Interprete Kernel}
In questa sezione, saranno dettagliate tutte le modifiche apportate all'interprete per renderlo compilabile ed eseguibile sulla scheda FPGA. 

\vspace{0.3cm}
\noindent Il codice completo relativo alla sezione seguente è presente nell'appendice (ref a vadd.cpp)
\vspace{0.3cm}

\noindent Finora, il codice dell'interprete sfrutta ampiamente i paradigmi offerti da un linguaggio di alto livello come il C.
Tuttavia va sottolineato che non è possibile utilizzare molti di questi paradigmi per sviluppare un kernel compilabile ed eseguibile su una scheda FPGA. Questo perché la generazione tramite High-Level Synthesis (HLS) messa a disposizione da Vitis, impone restrizioni specifiche che devono essere rispettate per completare il processo di compilazione del kernel e garantire il suo corretto funzionamento.

\vspace{0.3cm}

Durante la conversione dell'interprete al paradigma di funzionamento di un kernel sono stati riscontrati due problemi. In primo luogo, non è permesso l'uso di doppi puntatori, e in secondo luogo, non è possibile utilizzare chiamate ricorsive. Questi vincoli hanno creato due problemi, il primo riguarda la dichiarazione delle istruzioni, le quali come già precedentemente descritto precedentemente (\ref{structinstr}), sono definite come un puntatore di puntatori a interi da \texttt{8 bit}. Questo è stato risolto dichiarando a priori la dimensione massima del vettore delle istruzioni (ovvero il programma assembler) e quindi utilizzando la definizione di una matrice invece che un doppio puntatore. Come si può vedere nella seguente maniera: 
\begin{lstlisting}[language=C]
#define MAX_INSTR 32
    
int8_t instr[MAX_INSTR][4];
\end{lstlisting}
Come si può notare, possiamo gestire fino 32 istruzioni, una quantità che comunque può essere cambiata, e che si è dimostrata più che sufficiente per gli scopi di test di questo interprete. 

\vspace{0.3cm}

Il problema delle chiamate ricorsive è stato risolto eliminando le istruzioni che facevano uso del delay slot. Questa decisione è stata presa perché non era un obbiettivo primario di questa tesi, e inoltre anche per mantenere l'implementazione relativamente semplice e focalizzata sugli aspetti essenziali del funzionamento di una FPGA.
Una possibile soluzione consiste nel aggiungere un parametro nella funzione \texttt{run\_instruction} (fig. \ref{runinstruction}). Questo parametro viene verificato ad ogni iterazione del ciclo dell'interprete (fig. \ref{ciclointerprete}), e se attivo, consente l'esecuzione dell'istruzione presente nel Delay Slot, per poi tornare al normale flusso di esecuzione.

\vspace{0.3cm}

\noindent Successivamente alla risoluzione di questi problemi, la procedura per sviluppare un kernel richiede la scrittura di una funzione che viene chiamata  all'avvio dell'esecuzione sulla FPGA. 
Procediamo con il far vedere la segnatura della funzione:
\begin{lstlisting}[language=C]
void interprete(struct Memory *mem, 
                struct Registers *reg, 
                int32_t *out, 
                ap_uint<32> my_size) 
{
   ...
}
\end{lstlisting}
Questa funzione prende come parametri la struttura della memoria, la struttura dei registri, un intero da \texttt{32 bit} chiamato \texttt{out} per restituire il risultato  e \texttt{my\_size} che rappresenta la grandezza delle istruzioni. Questa scelta di utilizzare un singolo valore per verificare il corretto funzionamento è stata fatta poiché ai fini dei test iniziali è stato sufficiente restituire il risultato dell'esecuzione salvato in un singolo registro.
Notare come in generale il risultato dell'esecuzione del codice sarà rappresentato nella memoria dati, che alla fine della computazione viene ricopiata nella memoria host, come vedremo nella sez. \ref{interpretegpu}
Notare che per ridurre la quantità di parametri, le istruzioni sono state caricate insieme alla memoria \texttt{mem}.
Notare che questi parametri sono dichiarati e inizializzati dal lato host, questo aspetto verrà spiegato nella sezione \ref{host}.

\vspace{0.3cm}

\noindent Appena all'inizio della funzione \texttt{interprete}, sono presenti le seguenti specifiche:

\begin{lstlisting}[language=C]
...
#pragma HLS INTERFACE m_axi port = mem bundle = gmem
#pragma HLS INTERFACE m_axi port = reg bundle = gmem
#pragma HLS INTERFACE m_axi port = out bundle = gmem
#pragma HLS INTERFACE ap_ctrl_hs port = return
...
\end{lstlisting}

Queste righe di codice sono delle direttive chiamate "HLS pragmas". Queste rappresentano delle specifiche HLS per il compilatore \texttt{v++}, utilizzato nella fase di sintesi hardware (\ref{compilazione}). Queste direttive specificano come avviene la creazione delle porte RTL, a partire dagli argomenti della funzione durante la sintesi dell'interfaccia. Queste porte rappresentano il punto di connessione tra l'hardware presente nel chip della FPGA e le strutture esterne, come in questo caso la memoria globale DDR presente nella scheda che ospita il chip dell'acceleratore.
Facendo cosi il tool HLS determina automaticamente i protocolli I/O usati per gestire lo scambio di informazioni tra la parte acceleratore e i gli altri componenti del sistema.

L'utilizzo di queste direttive consente di determinare automaticamente i protocolli I/O utilizzati per la gestione dello scambio di informazioni tra la parte acceleratore e gli altri componenti. In questo caso specifico (come specificato nella documentazione \cite{sitoDocumentazionePragma}), le prime tre direttive \texttt{m\_axi} definiscono le interfacce di tipo master AXI4 (Advanced Extensible Interface), per le strutture \texttt{mem}, \texttt{reg} e \texttt{out}. Tutte queste porte sono tutti blocchi AXI, hanno tutti un bit che dice quando la computazione è terminata, la quarta direttiva \texttt{ap\_ctrl\_hs)}, specifica di impostare a 1 il bit del blocco quando si esegue il comando \texttt{return}.

\vspace{0.3cm}

Successivamente alle direttive \texttt{\#pragma} nella funzione è presente il seguente codice:
\begin{lstlisting}[language=C,caption={Funzione Interprete},label={funzioneinterprete}]
...
  struct Registers reg_copy;
  struct Memory mem_copy;
  
  // Copia dei registri dalla memoria globale alla memoria locale
  for (int i = 0; i < 32; i++)
    reg_copy.r[i] = reg->r[i];
  reg_copy.c = reg->c;
  reg_copy.pc = reg->pc;
  reg_copy.im = reg->im;

  // Copia della memoria dalla memoria globale alla memoria locale
  for (int i = 0; i < 1024; i++)
    mem_copy.data[i] = mem->data[i];

  for (int i = 0; i < MAX_INSTR; i++)
    for (int j = 0; j < 4; j++)
        mem_copy.instr[i][j] = mem->instr[i][j];
        
  // Creazione dei puntatori per accedere alle copie locali      
  struct Registers *reg_copy_pointer = &reg_copy;
  struct Memory *mem_copy_pointer = &mem_copy;

  // Ciclo interprete
  while (reg_copy_pointer->pc < my_size)
    run_instruction(mem_copy_pointer->instr[reg_copy_pointer->pc], 
        mem_copy_pointer, 
        reg_copy_pointer, 
        mem_copy_pointer->instr, 
        false);
        
  // Restituzione del risultato al lato host
  *out = reg_copy_pointer->r[1];
}
\end{lstlisting}
Come è possibile notare dall'implementazione, viene eseguita una copia locale dei parametri provenienti dal lato host e caricati sulla DDR attraverso strutture allocate staticamente, cosi facendo queste strutture dati risulteranno allocate nei blocchi RAM interni alla FPGA stessa (come sarà dimostrato nella sez. \ref{utilizzofpga}). Questo per evitare errori generati a tempo di esecuzione del kernel sulla FPGA, Successivamente le copie locali dei parametri, ovvero \texttt{reg\_copy} e \texttt{mem\_copy}, vengono usate tramite puntatori al fine di mantenere la leggibilità del codice e la coerenza. Successivamente è presente un ciclo, che come già precedentemente spiegato \ref{Interprete Softcore} permette  di proseguire fino a quando il registro program counter non raggiunge l’ultima istruzione della lista.

Alla fine del ciclo viene eseguita una copia del registro \texttt{r1} sul parametro esterno \texttt{out} per restituire il valore del risultato della computazione.

Come già specificato precedentemente, il resto del codice rispetto alla versione spiegata nella sezione \ref{Interprete Softcore}, rimane invariato.

\section{Interfaccia Host}
\label{host}
In questa sezione saranno dettagliate le componenti e il loro funzionamento dell'interfaccia usata sulla macchina host per gestire il trasferimento dei dati e la computazione della scheda FPGA.

\vspace{0.3cm}

\noindent Il codice completo relativo alla sezione seguente è presente nell'appendice (\ref{codicehost1cu})

\vspace{0.3cm}

\noindent In questa sezione del codice, avvengono tutte le chiamate OpenCl per interfacciarsi con l'acceleratore FPGA. 
Il funzionamento generale è riassunto nel seguente workflow: 
\begin{enumerate}
    \item Acquisire il file \texttt{.xclbin}, risultante dalla sintesi hardware del kernel dell'interprete (\ref{compilazione}) e il file \texttt{.text} derivante dall'estrazione del bytecode da un programma scritto in assembler (\ref{estrazioneBytecode}).
    \item Preparare l'ambiente di esecuzione e caricare i dati nella memoria DDR della FPGA.
    \item Procedere con l'esecuzione della computazione all'interno dell'acceleratore FPGA.
    \item In fine estrarre i risultati dalla memoria DDR e restituirli.
\end{enumerate}

\vspace{0.3cm}

Per capirne il funzionamento iniziamo analizzando la porzione iniziale di codice dove vengono istanziate tutte le variabili necessarie per questa procedura:

\begin{lstlisting}[language=C++]
int main(int argc, char **argv)
{
	// Verfica numero argomenti
	if (argc != 2)
	{
		std::cout << "Usage: " << argv[0] << " <XCLBIN File>" << std::endl;
		return EXIT_FAILURE;
	}

	// Dichiarazione Variabili Opencl
	cl_int err;
	cl::CommandQueue q;
	cl::Context context;
	cl::Kernel krnl;
 
	bool valid_device = false;

	// Allocazione spazio risultato
	int32_t *result = (int32_t*)malloc(sizeof(int32_t));
	*result = 0;

	// Lettura dispositvo 
	auto devices = xcl::get_xil_devices();

	// Creazione binario OpenCL 
	std::string binaryFile = argv[1];
	auto fileBuf = xcl::read_binary_file(binaryFile);
	cl::Program::Binaries bins{{fileBuf.data(), fileBuf.size()}};
...
\end{lstlisting}
Si nota che il file \texttt{.xclbin} viene passato come argomento alla funzione \texttt{main} all'avvio dell'eseguibile.
Successivamente, vengono dichiarate le variabili OpenCL necessarie, tra cui:
\begin{itemize}
    \item \texttt{cl\_int err;}: utilizzata per la  verifica dei possibili errori derivanti dalle chiamate di funzione OpenCL.
    \item \texttt{cl::CommandQueue q}: rappresenta una coda dei comandi. In OpenCL una variabile di questo tipo viene utilizzata per inoltrare dei comandi all'acceleratore (FPGA in questo caso) per l'esecuzione di operazioni.
    \item \texttt{cl::Context context}: rappresenta il contesto OpenCL, ovvero un oggetto dove è possibile creare e gestire la memoria dedicata per l'acceleratore, caricare i kernel da eseguire, e  altro ancora.
    \item \texttt{cl::Kernel krnl}: rappresenta un kernel OpenCL, ovvero il programma che sarà eseguito dall'acceleratore nel formato binario specifico del dispositivo.
\end{itemize}

In seguito, osserviamo alcune chiamate di funzione dalla libreria \texttt{xcl}, il quale è il runtime driver fornito da Xilinx. 
In generale il loro funzionamento è il seguente:
\begin{itemize} 
    \item \texttt{xcl::get\_xil\_devices()}: restituisce una lista di dispositivi compatibili con la piattaforma Xilinx.
    \item \texttt{xcl::read\_binary\_file(binaryFile)} restituisce un buffer a partire dal contenuto del file \texttt{.xclbin}.
\end{itemize}

Successivamente viene creato un oggetto "\texttt{cl::Program::Binaries bins}", il quale rappresenta il contenuto binario di un programma OpenCL.

Di seguito il codice per effettuare la programmazione dei dispositivi di accelerazione:
\begin{lstlisting}[language=C++,caption={ricerca dispositi},label={lst:ricerca}]
...
 for (unsigned int i = 0; i < devices.size(); i++)
 {
    auto device = devices[i];   
    
    // Creazione della coda di comando e del contesto per i device presenti
    OCL_CHECK(err, context = cl::Context(device, nullptr, nullptr, nullptr, &err));
    OCL_CHECK(err, q = cl::CommandQueue(context, device, CL_QUEUE_PROFILING_ENABLE, &err)); 
    std::cout << "Trying to program device[" << i << "]: " << device.getInfo<CL_DEVICE_NAME>() << std::endl;
    cl::Program program(context, {device}, bins, nullptr, &err);    
    
    if (err != CL_SUCCESS)
    {
        std::cout << "Failed to program device[" << i << "] with xclbin file!\n";
    }
    else
    {
        std::cout << "Device[" << i << "]: program successful!\n"; 
        
        //Creazione Kernel
        OCL_CHECK(err, krnl = cl::Kernel(program, "interprete", &err)); 
        valid_device = true;
        break; // Device valido trovato
    }
 }
...
\end{lstlisting}
Possiamo osservare come per ogni dispositivo (nel nostro caso è sempre stato solo uno, ovvero l'FPGA), vengono creati il contesto e la coda di comando utilizzando le funzioni di supporto di OpenCL. Inoltre viene creato l'oggetto \texttt{program}, , il quale rappresenta il programma che sarà eseguito nell'FPGA. Una volta creato il programma, viene utilizzato per creare il kernel, nel quale durante la creazione, specifichiamo il nome della funzione che verrà chiamata all'avvio dell'esecuzione.
Notare come eventuali errori vengo gestiti e registrati nella variabile \texttt{err}.

Successivamente vengono caricate le istruzioni dal file \texttt{.text}, vengono inizializzati memoria e registri dell'interprete, e le istruzioni vengono copiate all'interno della memoria dati. Di seguito è riportato il codice corrispondente: 

\begin{lstlisting}[language=C++]
...
 int32_t instr_size = 0;
 int8_t **instr_vector = get_instructions_from_file(file, &instr_size);
 auto mysize = instr_size;

 struct Memory *data = (struct Memory *)malloc(sizeof(struct Memory));
 struct Registers *reg = (struct Registers *)malloc(sizeof(struct Registers));

 // Inizializzazione Registri
 reg = inizialize_registers(reg);

 // Inizializzazione Memoria
 for (int i = 0; i < 1024; i++)
    data->data[i] = 0;

 // Copia istruzioni
 for (int i = 0; i < mysize; i++)
 {
    data->instr[i][0] = instr_vector[i][0];
    data->instr[i][1] = instr_vector[i][1];
    data->instr[i][2] = instr_vector[i][2];
    data->instr[i][3] = instr_vector[i][3];
 }
 ...
\end{lstlisting}

\vspace{0.3cm}

\noindent Successivamente si trova l'ultima sezione del codice dell'interfaccia host, dove continua la gestione della memoria e inizia la computazione dell'acceleratore, di seguito l'implementazione di questa: 
\begin{lstlisting}[language=C++]
...
 // Allocazione Buffer nella memoria globale della FPGA
 OCL_CHECK(err, cl::Buffer buffer_out(context, CL_MEM_USE_HOST_PTR | CL_MEM_WRITE_ONLY, sizeof(int32_t), result, &err));
 OCL_CHECK(err, cl::Buffer buffer_data(context, CL_MEM_USE_HOST_PTR | CL_MEM_READ_ONLY, sizeof(struct Memory), data, &err));
 OCL_CHECK(err, cl::Buffer buffer_reg(context, CL_MEM_USE_HOST_PTR | CL_MEM_READ_ONLY, sizeof(struct Registers), reg, &err));

 // Configurazione degli argomenti del kernel
 OCL_CHECK(err, err = krnl.setArg(0, buffer_data));
 OCL_CHECK(err, err = krnl.setArg(1, buffer_reg));
 OCL_CHECK(err, err = krnl.setArg(2, buffer_out));
 OCL_CHECK(err, err = krnl.setArg(3, mysize));

 // Migrazione dei dati dalla memoria host alla memoria globale della FPGA
 OCL_CHECK(err, err = q.enqueueMigrateMemObjects({buffer_data, buffer_reg}, 0 /*from host*/));

 // Inizio esecuzione kernel
 OCL_CHECK(err, err = q.enqueueTask(krnl));

 // Migrazione risultati dall'acceleratore alla memoria host 
 OCL_CHECK(err, err = q.enqueueMigrateMemObjects({buffer_out}, CL_MIGRATE_MEM_OBJECT_HOST));

 // Attesa che il kernel completi l'esecuzione
 OCL_CHECK(err, err = q.finish());

 std::cout << "\n\n FPGA RESULT " << *result << "\n\n";

 return EXIT_SUCCESS;
}
\end{lstlisting}
Si nota come inizialmente, si creano oggetti di tipo \texttt{cl::Buffer}, i quali rappresenta buffer di memoria sulla FPGA. Successivamente, gli argomenti della funzione iniziale presente nel kernel vengono configurati correttamente, in modo da fornire gli input nell'ordine appropriato. Viene quindi utilizzata la funzione \texttt{enqueueMigrateMemObjects} per aggiungere alla coda di comandi il comando che migra i dati dai buffer \texttt{buffer\_data} e \texttt{buffer\_reg} dalla memoria host alla memoria della FPGA (notare come l'argomento 0 specifica la direzione della migrazione). Successivamente, viene aggiunto alla coda di comando per avviare l'esecuzione del kernel.
L'istruzione \texttt{q.finish()} è utilizzata per attendere che tutte le operazioni all'interno della coda finiscano prima di proseguire con il resto del codice.
Una volta eseguite le operazione il risultato viene stampato e il programma termina.

\clearpage

\section{Compilazione}
\label{compilazione}
Vitis mette a disposizione 3 modi di compilare il progetto tramite il compilatore da loro offerto \texttt{v++}:

\begin{itemize}
    \item \textbf{Software Emulation}: in questa modalità, il kernel è compilato per essere eseguito sulla CPU della macchina host. Questo processo permette di rifinire facilmente il codice attraverso step iterativi di cicli build-and-run.
    \item \textbf{Hardware Emulation}: in questa modalità, il kernel viene compilato nel modello hardware (RTL), il quale viene eseguito in emulatore dedicato. Sebbene il processo di compilazione richieda più tempo, fornisce dettagli più approfonditi sull'esecuzione del kernel, e delle performance.
    \item \textbf{Hardware}: in questo processo il kernel viene compilato nell'hardware model (RTL) e viene implementato direttamente nella FPGA.
\end{itemize}
    
La prima fase del processo di compilazione, come indicato nella documentazione \cite{sitoDocumentazioneVitis}, consiste nel configurare l'ambiente per eseguire Vitis. Si utilizzano i seguenti comandi: 
\begin{lstlisting}[language=Bash]
source /tools/Xilinx/Vitis/2023.1/settings64.sh
source /opt/xilinx/xrt/setup.sh
\end{lstlisting}

Successivamente è necessario configurare correttamente la seguente variabile di ambiente: 
\begin{lstlisting}[language=Bash]
export PLATFORM_REPO_PATHS=/opt/xilinx/platforms/xilinx_u50_gen3x16_xdma_5_202210_1/
\end{lstlisting}
Questa variabile di ambiente indica il percorso per le piattaforme istallate sulla macchina corrente. Possiamo notare come la nostra piattaforma sia la scheda \texttt{u50}.

Successivamente il processo continua tramite i seguenti comandi:
\begin{lstlisting}[language=Bash]
export XCL_EMULATION_MODE=sw_emu
g++ -g -std=c++17 -Wall -O0 src/host.cpp src/xcl2.cpp -o ./app.exe -I$XILINX_XRT/include/ -L$XILINX_XRT/lib -lxrt_coreutil -pthread -lOpenCL
emconfigutil --platform xilinx_u50_gen3x16_xdma_5_202210_1
v++ -c -t sw_emu --platform xilinx_u50_gen3x16_xdma_5_202210_1 -k vadd -I/src src/vadd.cpp -o ./vadd.xo
v++ -l -t sw_emu --platform xilinx_u50_gen3x16_xdma_5_202210_1 ./vadd.xo -o ./vadd.xclbin
\end{lstlisting}
È necessario far notare come:
\begin{itemize}
    \item Vediamo come inizialmente impostiamo il metodo di emulazione.
    \item Il compilatore \texttt{g++} (come si può vedere nella fig. \ref{funzionamentoOpenCL} e offerto da GNU \cite{sitognu}), è utilizzato per compilare la sezione del progetto, che racchiude l'interfaccia lato host spiegata nella sez. \ref{host}.
    \item \texttt{emconfigutil} genera un file di configurazione che definisce il tipo e la quantità dei dispositivi da emulare per la piattaforma specificata in precedenza.
    \item \texttt{v++ -c}: compila il codice sorgente del kernel, producendo il file \texttt{.xo}.
    \item \texttt{v++ -l}: effettua il link del kernel compilato con la piattaforma di destinazione e genera il file \texttt{.xclbin}.
\end{itemize}

Successivamente per avviare il programma (compilato in \texttt{sw\_emu}) è sufficiente il seguente comando: 
\begin{lstlisting}[language=Bash]
./app.exe vadd.xclbin
\end{lstlisting}

Nel caso di voler usare la \texttt{hw\_emu} è sufficiente sostituire le seguenti righe:
\begin{lstlisting}[language=Bash]
export XCL_EMULATION_MODE=hw_emu
v++ -c -t hw_emu --platform xilinx_u50_gen3x16_xdma_5_202210_1 -k vadd -I/src src/vadd.cpp -o ./vadd.xo
v++ -l -t hw_emu --platform xilinx_u50_gen3x16_xdma_5_202210_1 ./vadd.xo -o ./vadd.xclbin
\end{lstlisting}
Notare che il resto dei comandi rimane invariato. Per la compilazione con target l'hardware effettivo non è più richiesta la variabile di ambiente \texttt{XCL\_EMULATION\_MODE}, e le seguenti righe diventano:
\begin{lstlisting}[language=Bash]
v++ -c -t hw --platform xilinx_u50_gen3x16_xdma_5_202210_1 -k vadd -I/src src/vadd.cpp -o ./vadd.xo
v++ -l -t hw --platform xilinx_u50_gen3x16_xdma_5_202210_1 ./vadd.xo -o ./vadd.xclbin
\end{lstlisting}

\clearpage



\section{Interprete Floating Point}
\label{interpretefloating}
In questa sezione verranno mostrati i cambiamenti apportati al codice dell'interprete spiegato nella sez. \ref{Interprete Kernel}, al fine di poter effettuare calcoli in virgola mobile.


Questi cambiamenti, come spiegato nella sez. \ref{funzionamento}, vengono resi possibili dal fatto che l'interprete è implementato in un linguaggio di alto livello.
E' importante notare che abbiamo simulato il funzionamento delle istruzioni finora implementate in virgola mobile, allo scopo di osservare i cambiamenti che questo comportava e cosa sarebbe successo al kernel compilato all'interno della FPGA. Sebbene il processore MicroBlaze disponga già di istruzioni per il calcolo in virgola mobile dei registri, abbiamo scelto di apportare solo cambiamenti "minimali" per restare coerenti con il titolo della tesi, e mantenere un approccio centrato su un "core GPU". Inoltre questo cambiamento della rappresentazione interna in virgola mobile non implica la possibilità di inserire valori floating point direttamente nel linguaggio assembly

Iniziamo con i cambiamenti immediati, i quali includono le seguenti modifiche nel codice:
\begin{lstlisting}[language=C]
 struct Registers
 {
    float r[32];
    ...
 };
 
 struct Memory
 {
    float data[1024];
    ...
 };

 struct Instruction
 {
    ...
    float im;
 };
\end{lstlisting}
Si nota come sia i registri che la memoria si dichiarano come  \texttt{float}, rappresentando cosi numeri con la virgola mobile da \texttt{32 bit}. Questo permette di eseguire le operazioni aritmetiche offerte dal linguaggio C per i numeri con la virgola.

Nelle istruzioni (\texttt{struct Instruction}) notiamo che la rappresentazione dell'immediato è in float.

Inoltre va notato che le operazioni bit a bit non sono direttamente supportate con i numeri in virgola mobile, quindi tutte le funzioni che usano questi operatori trattano il dato come un tipo \texttt{unsigned int} da 32 bit . Questo approccio garantisce la correttezza del codice in caso di operazioni bit a bit, di seguito un esempio della modifica all'istruzione \texttt{AND} seguendo questo paradigma:
\begin{lstlisting}[language=C]
  case 0x21:
  { /* AND 100001 */
    instr = parse_instruction(instruction, TYPE_A, instr, &reg->im);
    reg->r[instr->rd] = u_to_f(f_to_u(reg->r[instr->ra]) & f_to_u(reg->r[instr->rb]));
    update_PC(reg, 4, delay);
    break;
  }
\end{lstlisting}
Possiamo notare come vengono usate le funzioni \texttt{f\_to\_u} e \texttt{u\_to\_f}, segue la loro implementazione: 

\begin{lstlisting}[language=C]
uint32_t f_to_u(float n)
{
	return *(uint32_t *)&n;
}

float u_to_f(uint32_t n)
{
	return *(float *)&n;
}
\end{lstlisting}

Queste funzioni effettuano una conversione sull'intera rappresentazione in bit del tipo passato come argomento. Il motivo dell'utilizzo di queste funzioni rispetto ad un normale cast, è che se abbiamo un valore di tipo \texttt{float} e quindi un numero in virgola mobile di \texttt{32 bit}, e vogliamo eseguire un cast a un \texttt{uint32\_t} (ovvero un \texttt{unsigned int} da \texttt{32 bit}), utilizzando la notazione \texttt{(uint32\_t)}, non tutti i \texttt{32 bit} del valore vengono utilizzati, ma si perdono i bit che rappresentano i numeri dopo la virgola. Visto che il nostro obbiettivo è reintrepretare da \texttt{float} a \texttt{uint32\_t} e viceversa utilizzando tutti i \texttt{32 bit}, usiamo queste due funzioni anziché il cast offerto dal linguaggio C.

\vspace{0.3cm}

Inoltre un'altra modifica significativa riguarda la funzione di gestione dell'overflow, di seguito l'implementazione:
\begin{lstlisting}[language=C]
float add_Check_Overflow(float a, float b, bool *c)
{
	double res = (double)a + (double)b; /* only last 32 bit*/
	if (res > INT32_MAX)
	{
		*c = true;
		res = INT32_MAX;
	}
	else if (res < INT32_MIN) /* underflow */
	{

		*c = true;
		res = INT32_MIN;
	}
	else
	{
		*c = false;
	}

	return (float)res;
}
\end{lstlisting}
Questa funzione, analogamente alla funzione precedente, usa la rappresentazione successiva più grande di un \texttt{float}, ovvero un \texttt{double}, per facilitare la possibilità di determinare la presenza di un overflow. Il massimo valore rappresentabile è rimasto il massimo dei numeri \texttt{signed int} per una questione di praticità.

Una volta implementati questi cambiamenti si dispone di un interprete che esegue operazioni rappresentando i bit passati attraverso le istruzioni assembler con la virgola mobile.

\section{Control Unit}
\label{ControlUnit}
In questa sezione, verrà illustrato il processo di creazione di più istanze, utilizzando il codice spiegato nella sez. \ref{Interprete Kernel}.
Queste "istanze", verranno eseguite separatamente per ogni Control Unit, le quali come possiamo vedere nella fig. \ref{funzionamento}, risiedono all'interno del chip FPGA, e funzionano come "core", i quali sono capaci di eseguire computazioni parallele, sia della solita istanza del kernel o anche di diverse a seconda della necessità.

Il processo è reso possibile dal file di configurazione \texttt{conf.cfg}, il quale passato come argomento al compilatore \texttt{v++} durante le fasi di compilazione e di linking, abilita questo tipo di configurazione. Di seguito le modifiche apportate al compilatore:

\begin{lstlisting}[language=Bash]
v++ -c -t hw --platform xilinx_u50_gen3x16_xdma_5_202210_1 --config conf.cfg -k vadd -I/src src/vadd.cpp -o ./vadd.xo
v++ -l -t hw --platform xilinx_u50_gen3x16_xdma_5_202210_1 --config conf.cfg ./vadd.xo -o ./vadd.xclbin
\end{lstlisting}
Notiamo che l'unica modifica consiste nell'aggiunta della flag  \texttt{--config} seguita dal percorso al file \texttt{conf.cfg}.

Il contenuto \texttt{conf.cfg} del file è il seguente:
\begin{lstlisting}[language=Bash]
debug=1
save-temps=1

[connectivity]
nk=interprete:4:interprete_1,interprete_2,interprete_3,interprete_4

[profile]
data=all:all:all
\end{lstlisting}
Questo file in sostanza fornisce ulteriori specifiche al compilatore, oltre quelle passate come argomento.

Le prima due righe abilitano la creazione del file di debug e il la conservazione dei file temporanei generati durante la fase di compilazione.

La sezione \texttt{profile} specifica di salvare tutte le metriche disponibili sulla memoria passata come argomento al kernel.

La sezione \texttt{connectivity} specifica quante istanze del kernel creare.

\noindent Tramite la seguente stringa andiamo a creare 4 istanze del kernel \texttt{interprete}, di nome \texttt{interprete\_1}, \texttt{interprete\_2} e cosi via:
\begin{lstlisting}[language=Bash]
nk=interprete:4:interprete_1,interprete_2,interprete_3,interprete_4
\end{lstlisting}

Una volta creato il file di configurazione e modificati i parametri di compilazione, l'operazione successiva consiste nell'adattare l'interfaccia host in modo che sia in grado di gestire più Control Unit.

Per raggiungere questo obbiettivo, è stato necessario evitare di ripetere il codice, creando una funzione per la gestione di tutto il processo di creazione e istanziazione della memoria per un singolo kernel. Prima di vedere l'implementazione di tale funzione, di seguito una modifica all'interno del codice listato nella fig. \ref{lst:ricerca}, dove si usa un array di kernel dichiarato nella seguente maniera: 

\begin{lstlisting}[language=C]
	cl::Kernel krnl[KERNEL_NUMBER];
	int32_t result[KERNEL_NUMBER];
\end{lstlisting}

Possiamo notare come all'aumentare del numero di Control Unit, aumenta di conseguenza il numero risultati che devono essere estratti dalla FPGA.

All'interno del ciclo di ricerca dispositivi listato in fig. \ref{lst:ricerca} viene usato il seguente approccio per configurare i kernel una volta trovato un dispositivo corretto: 
\begin{lstlisting}[language=C]
    ...
		else
		{
			std::cout << "Device[" << i << "]: program successful!\n";
			for (int i = 1; i <= KERNEL_NUMBER; i++)
			{
				std::string var = "interprete:{interprete_" + std::to_string(i) + "}";
				OCL_CHECK(err, krnl[i - 1] = cl::Kernel(program, var.c_str(), &err));
			}
			valid_device = true;
			break;
		}
    ...
\end{lstlisting}
Possiamo vedere come si usi un approccio modulare per istanziare i kernel con la forma \texttt{nome kernel: nome CU}.

Successivamente, segue l'implementazione della funzione che gestisce il singolo kernel per essere correttamente inserito nella coda di comando:
\begin{lstlisting}[language=C,caption={enqueue task},label={enqueuetask}]
void enqueue_task(char const *file,
									cl::CommandQueue *q,
									cl::Kernel krnl,
									cl::Context context,
									int32_t *result)
{
	cl_int err;
	int32_t instr_size = 0;
	int8_t **vector = get_instructions_from_file(file, &instr_size);
	auto mysize = instr_size;
	struct Memory *data = (struct Memory *)malloc(sizeof(struct Memory));
	struct Registers *reg = (struct Registers *)malloc(sizeof(struct Registers));
	reg = inizialize_registers(reg);

	for (int i = 0; i < 1024; i++)
	{
		data->data[i] = 0;
	}

	for (int i = 0; i < mysize; i++)
	{
		data->instr[i][0] = vector[i][0];
		data->instr[i][1] = vector[i][1];
		data->instr[i][2] = vector[i][2];
		data->instr[i][3] = vector[i][3];
	}

	OCL_CHECK(err, cl::Buffer buffer(context, CL_MEM_USE_HOST_PTR | CL_MEM_WRITE_ONLY, sizeof(int32_t), result, &err));

	OCL_CHECK(err, cl::Buffer buffer_data(context, CL_MEM_USE_HOST_PTR | CL_MEM_READ_ONLY, sizeof(struct Memory), data, &err));

	OCL_CHECK(err, cl::Buffer buffer_reg(context, CL_MEM_USE_HOST_PTR | CL_MEM_READ_ONLY, sizeof(struct Registers), reg, &err));

	OCL_CHECK(err, err = krnl.setArg(0, buffer_data));
	OCL_CHECK(err, err = krnl.setArg(1, buffer_reg));
	OCL_CHECK(err, err = krnl.setArg(2, buffer));
	OCL_CHECK(err, err = krnl.setArg(3, mysize));

	OCL_CHECK(err, err = q->enqueueMigrateMemObjects({buffer_data, buffer_reg}, 0 /* 0 means from host*/));

	OCL_CHECK(err, err = q->enqueueTask(krnl));

	OCL_CHECK(err, err = q->enqueueMigrateMemObjects({buffer}, CL_MIGRATE_MEM_OBJECT_HOST));
}
\end{lstlisting}
La funzione accetta come argomenti il nome del file \texttt{.text}, che contiene la lista di istruzione dell'interprete, la coda di comando \texttt{q}, il kernel precedentemente creato, il contesto e la variabile risultato.
Il resto del codice di questa funzione è stato spiegato nella sezione precedente ovvero sez. \ref{host}.

Successivamente al ciclo di ricerca di dispostivi, abbiamo il seguente codice, dove grazie alla funzione precedentemente spiegata ovvero \texttt{enqueue\_task} passiamo tutti i comandi in ordine di kernel alla coda di comando, attendiamo che questa finisca il lavoro, e andiamo a stampare i risultati.

\begin{lstlisting}[language=C]
...
	char const *file[KERNEL_NUMBER] = {
			"data.text",
			"add.text",
			"bitop.text",
			"sub.text",
	};


	for (int i = 0; i < KERNEL_NUMBER; i++)
	{
		result[i] = 0;
		enqueue_task(file[i], &q, krnl[i], context, &result[i]);
	}

	OCL_CHECK(err, err = q.finish());

	for (int i = 0; i < KERNEL_NUMBER; i++)
		std::cout << "\nKernel " << i << "  result:" << *result;

  std::cout << "\n";
	return EXIT_SUCCESS;
}
\end{lstlisting}
In questo modo abbiamo eseguito con successo 4 kernel distinti, ognuno con un file di istruzioni assembler diverso, all'interno della FPGA.

\section{Simulazione GPU con codice MIMD}
\label{interpretegpu}


In questa sezione illustreremo la configurazione e le modifiche apportate all'interprete del softcore e all'interfaccia host, per simulare una GPU con core indipendenti, senza controllore SIMD. Con questo approccio si vuole mostrare come più istanze/CU (Control Unit) del softcore all'interno della FPGA possano eseguire computazioni in modo data parallel, ma con codici distinti per ciascuna (quindi MIMD nel senso che sullo stesso dato a posizione diverse vengono effettuati calcoli diversi). In particolare, si potrà eseguire del codice "quasi SIMD", dove la stessa funzione \texttt{f} viene applicata su tutti i \texttt{v[i]} (il dato in posizione \texttt{i} del vettore \texttt{v} su cui applicare la funzione), ma con dei branch interni alla funzione \texttt{f}, i quali dipendono dal singolo dato \texttt{v[i]}. 


\vspace{0.3cm}
In sostanza, oltre al fatto che le Control Unit sono diventate 8 invece che 4 come nella sezione \ref{ControlUnit}, e che il risultato in questa sezione diventi tutta la memoria dell'interprete invece che restituire come risultato della computazione un intero da \texttt{32 bit}, i cambiamenti sono i seguenti:

All'interno del kernel la lista di parametri e il modo in cui restituiamo il risultato cambiano come segue:
\begin{lstlisting}[language=C]
void vadd(struct Memory *mem, struct Registers *reg, struct Memory *outmem, ap_uint<32> my_size)
{
...
...
    for (int i = 0; i < 1024; i++)
        outmem->data[i] = mem_copy.data[i];
 }
}
\end{lstlisting}

Inoltre è cambiata l'interfaccia lato host per gestire questa  modifica. Di seguito è riportata l'implementazione:
\begin{lstlisting}[language=C]
...
	struct Memory *data_out[KERNEL_NUMBER];

	for (int i = 0; i < KERNEL_NUMBER; i++)
	{
		data_out[i] = (struct Memory *)malloc(sizeof(struct Memory));
		enqueue_task(file[i], &q, krnl[i], context, data_out[i]);
	}

	OCL_CHECK(err, err = q.finish());

	for (int i = 0; i < KERNEL_NUMBER; i++)
	{
		print_vector(15, data_out[i], 1);
		free(data_out[i]);
	}

	return EXIT_SUCCESS;
}
\end{lstlisting}

Il codice per il resto è rimasto praticamente invariato, con la principale modifica che riguarda gli argomenti di \texttt{enqueue\_task}. Dove invece di gestire e caricare un intero gestiamo e carichiamo la memoria \texttt{data\_out} all'interno del kernel. Inoltre al termine della computazione e del trasferimento dei file dei dati di tutti i kernel, vengono stampati i primi 15 elementi della memoria per valutare i risultati dell'esecuzione, e quindi il programma termina.
\chapter{Risultati}
In questo capitolo affronteremo il processo di verifica dei risultati per le diverse versioni dell'interprete spiegate nella sez.\ref{implementazione}. Seguiranno una sezione sulla verifica del funzionamento di ciascuna versione, una sezione successiva per l'analisi sull'occupazione della FPGA per ogni versione dell'interprete, e infine concluderemo con una vista sull'esecuzione a run time dei vari kernel attraverso gli strumenti di Vitis.

\section{Verifica Funzionamento}
La verifica del funzionamento delle diverse versioni del codice è stata svolta utilizzando dei file in codice assembler per il processore Microblaze. Questi file sono stati eseguite sull'interprete implementato nella sez. \ref{Interprete Softcore}, eseguito sulla macchina host. Questi risultati sono stati confrontati con i risultati estratti dalle versioni compilate ed eseguite sul FPGA per verificare il corretto funzionamento dell'interprete.

Partiamo mostrando come i risultati sono stati prima verificati sulla macchina host.
E' stato creato un file \texttt{test.c}, il quale per ogni file assembler di test, avvia l'interprete e confronta i risultati, la funzione utilizzata all'interno del file è la seguente:
\begin{lstlisting}[language=C]
bool test (char* file,int32_t n) 
{
    int32_t instructions_size = 0;
    int8_t **instructions = get_instructions_from_file(file, &instructions_size);

    struct Memory *data = malloc(sizeof(struct Memory));
    data = inizialize_memory(MEMORY_SIZE, data);

    struct Registers *reg = malloc(sizeof(struct Registers));
    reg = inizialize_registers(reg);

    while (reg->pc < instructions_size)
        run_instruction(instructions[reg->pc], data, reg, instructions, false);

    if (reg->r[1] == n)
        return true;
        
    return false;
}
\end{lstlisting}

Possiamo notare come la funzione prende in input il nome del file contenente le istruzioni, e un valore intero \texttt{n}, ovvero il risultato atteso dalla computazione. 
Dopo aver inizializzato tutto il necessario per avviare l'interprete sulla macchina host, la funzione avvia il ciclo dell'interprete. Al termine confronta il risultato ottenuto con quello atteso e restituisce un valore booleano.

\vspace{0.3cm}

\noindent Questa funzione all'interno del file \texttt{test.c} viene utilizzata nel seguente modo:
\begin{lstlisting}[language=C]
int main () 
{
    if (test("test/add.text",ADD_RESULT)) 
        printf("ADD test OK\n");
    else 
        printf("ADD test FAIL\n");   
        
    if (test("test/sub.text",SUB_RESULT)) 
        printf("SUB test OK\n");
    else 
        printf("SUB test FAIL\n");   

    if (test("test/bitop.text",BIT_OP_RESULT)) 
        printf("BIT_OP test OK\n");
    else 
        printf("BIT_OP test FAIL\n");   

    if (test("test/branch.text",BRANCH_RESULT)) 
        printf("BRANCH test OK\n");
    else 
        printf("BRANCH test FAIL\n");   

    if (test("test/data.text",DATA_RESULT)) 
        printf("DATA test OK\n");
    else 
        printf("DATA test FAIL\n");   

    return EXIT_SUCCESS;
}
\end{lstlisting}
I file mostrati di seguito sono disponibili nell'appendice alla sez. \ref{fileassembler}. I risultati specificati nei define del codice sono i seguenti:

\begin{lstlisting}[language=C,caption={Risultati Aspettati},label={risultatiaspettati}]
#define ADD_RESULT 159
#define SUB_RESULT -2147482620
#define BIT_OP_RESULT 200
#define BRANCH_RESULT 36
#define DATA_RESULT 199
\end{lstlisting}
Si noti che questi risultati sono stati calcolati manualmente al fine di avere una verifica corretta:

Segue un esempio per il file \texttt{branch.text}:

\begin{lstlisting}
	.text
	.align	2
	.globl	main
	.ent	main
	.type	main, @function    

main:
	addi	r3,r0,8       
	addi	r2,r0,-8
	addi	r1,r0,0
	addi 	r1,r1,1
	addi 	r3,r3,-1
	bge r3,r2
	nop
	addi	r3,r0,8
	addi 	r1,r1,1
	addi 	r3,r3,-1
	bgei r3,-8
	nop
	addi	r3,r0,8
	addi 	r1,r1,1
	addi 	r3,r3,-1
	bgei r3,-8
	nop
	addi	r3,r0,-8
	addi	r2,r0,-8
	addi 	r1,r1,1
	addi 	r3,r3,1
	ble r3,r2  #r1 = 37+9 = 36 

	.end	main
\end{lstlisting}
Questo file ha come risultato il valore 36 nel registro \texttt{r1}.

\clearpage

Dopo aver compilato ed eseguito l'interprete, è evidente notare come i risultati ottenuti siano coerenti con quelli aspettati. Di seguito l'immagine della stampa dell'esecuzione:

\begin{figure}[h!]
\centering
\includegraphics[scale=0.35]{images/Capitolo5/1_im.png}
\caption{Interprete Softcore Risultati}
\label{1curisultati}
\end{figure}

\vspace{0.3cm}

Adesso consideriamo l'esecuzione dell'interprete spiegato nella sez. \ref{Interprete Kernel}, e quindi una singola control unit eseguita all'interno della FPGA.
Per mostrare il risultato è stato cambiato leggermente il codice per passare come argomento al main il nome del file assembler da eseguire.
Possiamo vedere che i risultati ottenuti combaciano con i risultati aspettati:

\begin{lstlisting}[language=C,caption={Risultato Branch}]
#define BRANCH_RESULT 36
\end{lstlisting}

\begin{figure}[h!]
\centering
\includegraphics[scale=0.35]{images/Capitolo5/2_im.png}
\caption{Interprete Kernel Risultati}
\label{typeB}
\end{figure}

Questo processo di verifica è stato ripetuto per tutti i file di test, confermando il corretto funzionamento del kernel implementato ed eseguito all'interno della FPGA.

\vspace{0.3cm}


Per la verifica della versione dell'interprete illustrata nella sez.\ref{ControlUnit}, ovvero 4 Control Unit, sono stati usati i file assembler precedentemente spiegati nella fig. \ref{risultatiaspettati}.In particolare, per ciascuna istanza del kernel istanziata, è stato usato un file distinto. 

\vspace{0.3cm}

\noindent Di seguito un immagine dell'esecuzione:

\begin{figure}[h!]
\centering
\includegraphics[scale=0.35]{images/Capitolo5/3_im.png}
\caption{Interprete 4 CU Risultati}
\label{4curisultati}
\end{figure}

\noindent Possiamo constatare che i risultati ottenuti combaciano con i risultati attesi, confermando la corretta esecuzione delle 4 istanze dell'interprete all'interno della FPGA che eseguono ciascuna un suo programma diverso dagli altri.

\vspace{0.3cm}

\noindent Per la verifica dell'interprete in virgola mobile illustrato nella sez. \ref{interpretefloating}, considerando che la rappresentazione dei valori in virgola mobile è solo interna all'interprete, per scrivere il file assembly utilizzato nei test, sono stati convertiti i numeri con la virgola usando la funzione di conversione \texttt{f\_to\_i}. Il funzionamento di questa funzione è lo stesso delle funzioni di conversione usate nella sez. \ref{interpretefloating}. Di seguito il programma scritto in C utilizzato per testare il risultato e i valori corretti da usare all'interno delle istruzioni assembler:

\begin{lstlisting}[language=C,label={codicecfloating},caption={Codice valori conversione floating}]
void main()
{
   float a = 177.13;
   printf("a: %d\n", f_to_i(a));

   float b = 231.9999;
   printf("b: %d\n", f_to_i(b));
   
   float c = a + b;

   float d = 99.3;
   printf("d signed: %d\n", f_to_i(d));
   float e = u_to_f(f_to_u(c) | f_to_u(d));

   float f = 20.8;
   printf("f signed: %d\n", f_to_i(f));
   float result = e - f;
   printf("res signed: %d\n", f_to_i(result));
   printf("res: %.6f\n", result);
}
\end{lstlisting}
Di seguito il codice assembler usato per il test dell'interprete in versione floating point: 

\begin{lstlisting}[caption={File Assembler per Test Floating Point},label={assemblerfloating}]
	.text
	.align	2
	.globl	main
	.ent	main
	.type	main, @function       

main:

	addi	r1,r0,1127293256      # r1 = 177.13
	addi	r2,r0,1130889209      # r2 = 231.9999
	add  r1,r1,r2               # r1 = 409.129883
	addi r3,r0,1120311706       # r3 = 99.3
	or r1,r1,r3                 # r1 = 413.200989
	addi r4,r0,1101424230	      # r4 = 20.8
	rsub r1,r4,r1	              # r1 = 392.401001

	.end	main
\end{lstlisting}
Possiamo notare come il file assembler listato in fig. \ref{assemblerfloating} utilizza le stesse operazioni del codice in C listato in fig. \ref{codicecfloating}. Il risultato dell'esecuzione è il valore in virgola mobile \texttt{392.401001}, il quale rappresentato come un \texttt{signed int} è \texttt{1136931668}. Di seguito l'esecuzione dell'interprete in versione Floating Point con il file assembler in listato in fig. \ref{assemblerfloating}:

\clearpage

\begin{figure}[h!]
\centering
\includegraphics[scale=0.35]{images/Capitolo5/11_im.png}
\caption{Interprete Floating Point Risultati}
\label{4curisultati}
\end{figure}


\vspace{0.3cm}

L'ultima versione dell'interprete da verificare è la quella illustrata della sez. \ref{interpretegpu}. Questo interprete, come precedentemente spiegato, restituisce l'intera memoria dati dopo l'esecuzione. Per questo, al fine di testare il corretto funzionamento di questo interprete, sono stati impiegati i seguenti file per ciascun "core". Di seguito è riportata la lista dei file utilizzati:  

\begin{lstlisting}[language=C]
char const *file[KERNEL_NUMBER] = {
		"gpu_1.text", // v[i] = v[i] + 1,
		"gpu_2.text", // +5
		"gpu_3.text", // -1
		"gpu_4.text", // *2
		"gpu_5.text", // *45
		"gpu_6.text", // & 45
		"gpu_7.text", // | 45
		"gpu_8.text", // >> 1
};
\end{lstlisting}

\vspace{0.3cm}
È evidente come per ciascuna Control Unit sia stato creato un file specifico in linguaggio assembly. Questo file prende un vettore di 15 interi dalla memoria dati, precedentemente inizializzato con numeri casuali. Per ciascun  elemento del vettore viene eseguita un'operazione diversa (come si può notare dai commenti affianco ai nomi dei file). Di seguito il codice assembly usato per compilare ed estrarre il file \texttt{gpu\_2.text}:

\begin{lstlisting}
	.text
	.align	2
	.globl	main
	.ent	main
	.type	main, @function      

main:

	addi	r1,r0,14
	addi	r8,r0,1 
	addi  r2,r0,0
	lwi	 r3,r2,0	
	addi r3,r3,5
	swi  r3,r2,0   
	rsub  r1,r8,r1
	addi  r2,r2,1
	bgei  r1,-20

	.end	main
\end{lstlisting}

Per inizializzare il vettore destinato ai test in memoria che sarà usato dal kernel, si parte dalla posizione \texttt{0} della memoria dati. Per eseguire questa operazione prima di caricare la memoria all'interno dell'interprete è stata apportata una modificata la funzione \texttt{enqueue\_task} spiegata nella fig. \ref{enqueuetask}, inserendo il seguente codice:

\begin{lstlisting}[language=C]
...
for (int i = 0; i < 15; i++)
{
	int32_t n = rand() % 1000;
	data->data[i] = n;
	vector_result[i] = n;
}
...
\end{lstlisting}

È possibile notare che per questioni di semplicità, i numeri si limitano ad un massimo di $1000$. Inoltre è evidente la presenza di un ulteriore vettore,  ovvero \texttt{vector\_result}, il quale viene inizializzato con gli stessi valori della memoria dati. Questo vettore viene calcolato dalla CPU della macchina host con le solite operazioni svolte da ogni istanza dell'interprete dentro la FPGA. Questa procedura ha lo scopo di confrontare il vettore con i risultati ottenuti in memoria per ogni kernel, in modo da verificare il corretto funzionamento. 

\vspace{0.3cm}

\noindent Di seguito è riportato il codice usato nell'interfaccia host per effettuare questa verifica: 

\begin{lstlisting}[language=C++]
...
	struct Memory *data_out[KERNEL_NUMBER];
	int vector_result[KERNEL_NUMBER][15];

	for (int i = 0; i < KERNEL_NUMBER; i++)
	{
		data_out[i] = (struct Memory *)malloc(sizeof(struct Memory));
		enqueue_task(file[i], &q, krnl[i], context, data_out[i], vector_result[i]);
	}

	OCL_CHECK(err, err = q.finish());

	std::cout << "\n";

	for (int i = 0; i < KERNEL_NUMBER; i++)
	{
		make_result(vector_result[i],i);
		compare(data_out[i],vector_result[i], i);
		free(data_out[i]);
	}

	return EXIT_SUCCESS;
}
\end{lstlisting}

Per ciascun kernel è presente un vettore risultati, i quali come spiegato precedentemente vengono inizializzati all'interno della funzione \texttt{enqueue\_task}. Successivamente attraverso la funzione \texttt{make\_result()}, vengono calcolati i valori lato host per ogni kernel seguendo le operazioni specificate nei file. In fine, per verificare che i risultati ottenuti siano in linea con i risultati attesi all'interno di \texttt{vector\_result}, viene utilizzata la funzione \texttt{compare}. Di seguito è riportata l'implementazione di tale funzione:

\begin{lstlisting}[language=C++]
void compare(struct Memory *data, int32_t *vector_result, int n)
{
	bool cmp = true;

	for (int i = 0; i<15; i++) 
		if (data->data[i] != vector_result[i])
			cmp = false;

	if (cmp) 
		std::cout << " Kernel: " << n << " Test Passed\n";
	else 
		std::cout << " Kernel: " << n << " Test Failed\n";
}
\end{lstlisting}
Possiamo notare come la funzione accetti come input la memoria restituita dall'interprete e il vettore dei risultati, calcolato sulla macchina host. Successivamente effettua un confronto tra i due e stampa il risultato del risultato ottenuto.

\vspace{0.3cm}

\noindent Di seguito è riportata l'immagine con il risultato dell'esecuzione del codice precedentemente descritto: 

\begin{figure}[h!]
\centering
\includegraphics[scale=0.45]{images/Capitolo5/4_im.png}
\caption{Interprete GPU Risultati}
\label{gpurisultati}
\end{figure}

\clearpage 

\section{Utilizzo FPGA}
\label{utilizzofpga}
Durante il processo di compilazione, il compilatore \texttt{v++} genera il file \texttt{.xclbin.link} \texttt{\_summary}. Questo  file contiene le informazioni sull'utilizzo effettivo delle risorse del chip della FPGA. Nel seguito di questa sezione verranno presentati i risultati ottenuti dalle diverse versioni dell'interprete. È importante notare che tutti i dati ottenuti durante questa fase siano stati letti utilizzando il software offerto da Vitis ovvero \texttt{vitis\_analyzer}. 

\vspace{0.3cm}

\noindent Di seguito i risultati relativi alla versione dell'interprete spiegata nella sez. \ref{Interprete Kernel}, corrispondente ad una singola istanza dell'interprete all'interno della FPGA:

\begin{figure}[h!]
\centering
\includegraphics[scale=0.40]{images/Capitolo5/5_im.png}
\caption{Utilizzo FPGA 1 Control Unit}
\label{utlizzo1cu}
\end{figure}

\vspace{0.3cm}

\noindent Nella tabella è possibile vedere i seguenti campi:

\begin{itemize}
    \item \textbf{LUT} (Look-Up Table): Gli elementi fondamentali per una FPGA per realizzare circuiti logici. Le LUT sono delle tabelle di ricerca usate per implementare funzioni logiche. Presentano un numero di ingressi fisso e generano un'uscita in base ai valori memorizzati, che variano a seconda della funzione che si vuole realizzare \cite{sitoLUT}.
    \item \textbf{LUTASMEM}: Sono una versione delle LUT che può funzionare come memoria lettura/scrittura.
    \item \textbf{REG} (Register): Nel contesto delle FPGA, i registri sono gruppi di flip-flop utilizzati per memorizzare i dati temporanei. Possono essere utilizzati per sincronizzare segnali, memorizzare risultati intermedi e altro ancora \cite{sitoREG}. 
    \item \textbf{BRAM} (Block RAM): Blocchi di memoria RAM configurabili all'interno della FPGA. Queste memorie sono utilizzate per memorizzare i dati in modo efficiente, migliorando cosi le prestazioni dei circuiti implementati per le FPGA.
    \item \textbf{URAM} (Ultra RAM): Blocchi di memoria RAM ad alte prestazioni progettate per massimizzare larghezza di banda e latenza. Questi tipi di RAM sono disponibili solo su alcune schede FPGA \cite{sitoURAM}.
    \item \textbf{DSP} (Digital Signal Processor): Sono delle ALU solitamente in Floating Point all'interno della FPGA dedicate alle per operazioni matematiche complesse come moltiplicazioni, accumuli e altro ancora \cite{sitoDSP}. 
\end{itemize}

\noindent È possibile notare come le righe della tabella in fig. \ref{utlizzo1cu}, riguardano la disponibilità totale delle risorse della piattaforma, quanto l'utente ha a disposizione in termini delle risorse totali e quanto il kernel compilato occupa di queste risorse.
È importante sottolineare che la quantità delle risorse disponibile dipende direttamente dal tipo di scheda utilizzata, nel nostro caso la Alveo U50 \ref{Specifiche-U50}.

\vspace{0.3cm}

\noindent Di seguito sono riportati i risultati dell'occupazione della FPGA per la versione dell'interprete spiegata nella sez. \ref{ControlUnit}, ovvero 4 istanze del kernel compilate all'interno della FPGA:

\begin{figure}[h!]
\centering
\includegraphics[scale=0.40]{images/Capitolo5/6_im.png}
\caption{Utilizzo FPGA 4 Control Unit}
\label{utilizzo4cu}
\end{figure}
 
Mettendo a confronto le tabelle in fig. \ref{utlizzo1cu} e in fig. \ref{utilizzo4cu} possiamo concludere che:

\begin{itemize}
    \item LUT: L'utilizzo è di circa $0.8 \%$ nella versione a 1 istanza del kernel, rispetto a circa $3.4 \%$ nella versione a 4.
    \item LUTASMEM: L'utilizzo è del $0.3 \%$ contro un $1 \%$ nella versione a 4 istanze del kernel.
    \item REG: L'utilizzo è del $0.6 \%$ dei registri rispetto a $2.5 \%$ per la versione a 4 kernel.
    \item BRAM: L' utilizzo è del $1 \%$ contro un $4 \%$ 
    \item URAM: L'utilizzo è del $0 \%$ in tutti e due casi.
    \item DSP: L'utilizzo è del $0.05 \%$ contro $0.2 \%$.
\end{itemize}
Va notato che le percentuali menzionate sono state approssimate per migliorare la leggibilità.
Da questi risultati emerge che, l'occupazione delle risorse cresce con l'istanziazione di più kernel all'interno della FPGA. Inoltre si osserva che le risorse utilizzate non aumentano esattamente in modo lineare con il numero di kernel, ma che una parte di risorse è impiegata anche per gestire la complessità introdotta dall'aumentare delle componenti hardware all'interno del chip della FPGA.

\vspace{0.3cm}

\clearpage

\noindent Di seguito sono riportati i risultati di occupazione della versione dell'interprete descritta in sez. \ref{interpretegpu}: 

\begin{figure}[h!]
\centering
\includegraphics[scale=0.40]{images/Capitolo5/7_im.png}
\caption{Utilizzo FPGA 8 Control Unit}
\label{utilizzo4cu}
\end{figure}

In questo caso notiamo un utilizzo del $7 \%$ delle LUT, $5 \%$ dei registri, e $12 \%$ di utilizzo delle BRAM. Questi risultati sono coerenti con il trend di crescita dell'utilizzo delle risorse osservato fin'ora. Inoltre va notato come con la scheda usata permetta di compilare ancora maggiore di kernel rispetto al massimo di 8 raggiunti nel codice citato in precedenza. Con  l'utilizzo delle risorse osservato finora, è possibile stimare che circa 56 kernel possono essere inseriti nell'FPGA utilizzata per gli esperimenti (la scheda Alveo U50 \ref{Alveo}).

\vspace{0.3cm}

\noindent Di seguito sono riportati i risultati dell'occupazione della FPGA per la versione dell'interprete spiegata nella sez. \ref{interpretefloating}, ovvero l'interprete con la rappresentazione interna dei valori in virgola mobile:

\begin{figure}[h!]
\centering
\includegraphics[scale=0.40]{images/Capitolo5/12_im.png}
\caption{Utilizzo FPGA Floating Point}
\label{utilizzo4cu}
\end{figure}

Possiamo notare, come, rispetto all'utilizzo della versione ad 1 CU senza virgola mobile, listata in fig. \ref{utlizzo1cu}, è presente un maggior utilizzo dei DSP, questo perché come spiegato precedentemente nella definizione delle DSP, queste ALU vengono usate per il calcolo in virgola mobile. Di conseguenza, come prevedibile, si è registrato un aumento del loro utilizzo rispetto alla versione senza virgola mobile.

\vspace{0.3cm}

È importante sottolineare che, durante lo sviluppo di questo interprete , l'ottimizzazione nell'occupazione delle risorse della FPGA non era un obbiettivo fondamentale. Nella documentazione del processore Microblaze \cite{sitoMicroblaze}, vengono mostrate delle statistiche sull'utilizzo delle risorse nelle varie schede FPGA del mercato con varie implementazioni del processore, sviluppate direttamente dagli sviluppatori del processore utilizzando un linguaggio RTL. Viene mostrato che il processore occupa mediamente circa 3000 LUT. Pertanto, possiamo concludere che anche se il nostro interprete implementi solo un sotto insieme di tutte le istruzioni assembler, e sia scritto in un linguaggio di alto livello e convertito in hardware tramite HLS offerto da Vitis, con un utilizzo  medio di 5600 LUT, possiamo considerarci soddisfatti, tenendo anche conto della facilità di configurazione e estensione che questa soluzione offre.

\vspace{0.3cm}

Durante lo sviluppo del progetto, è stata prestata abbastanza attenzione all'utilizzo delle BRAM all'interno della FPGA. Questo è stato fatto al fine di minimizzare l'uso della memoria globale DDR, la quale trovandosi fuori dal chip FPGA, aggiunge latenza e e richiede circuiti hardware aggiuntivi per gestire il processo di comunicazione.
Per verificare l'effettivo posizionamento dentro la BRAM delle strutture dati utilizzate all'interno del interprete, sono state sviluppate due versioni. In entrambe è presente un sottoinsieme ancora più piccolo di istruzioni per questioni di semplicità, ma in una versione per la computazione si utilizzavano direttamente i parametri passati dall'OpenCl e caricati nella memoria DDR. Nell'altra versione, si è seguito il paradigma illustrato nella sez. \ref{Interprete Kernel} e listato nella fig. \ref{funzioneinterprete} dove i parametri sono copiati su strutture dati dichiarate all'interno della funzione.

Di seguito il codice della versione che non effettua la copia locale dei parametri:

\begin{lstlisting}[language=C]
void interprete(struct Memory *mem, struct Registers *reg, uint32_t *out, ap_uint<32> my_size)
{
pragma HLS INTERFACE m_axi port = mem bundle = gmem
pragma HLS INTERFACE m_axi port = reg bundle = gmem
pragma HLS INTERFACE m_axi port = out bundle = gmem
pragma HLS INTERFACE ap_ctrl_hs port = return

	while (reg->pc < my_size)
		run_instruction(mem->instr[reg->pc], 
						mem, 
						reg, 
						mem->instr, 
						false);

	out[1] = reg->r[1];
}
\end{lstlisting}

In questa implementazione, si noti l'uso diretto dei parametri istanziati dalla parte host e caricati nella DDR.

\vspace{0.3cm}

Di seguito i risultati di occupazione delle due versioni:

\begin{figure}[h!]
\centering
\includegraphics[scale=0.40]{images/Capitolo5/9_im.png}
\caption{Utilizzo senza copia locale dei parametri}
\label{nopragma}
\end{figure}

\begin{figure}[h!]
\centering
\includegraphics[scale=0.40]{images/Capitolo5/8_im.png}
\caption{Utilizzo con copia locale dei parametri}
\label{pragma}
\end{figure}

È possibile notare come l'utilizzo delle BRAM sale da 1 a 16 con l'implementazione che prevede la copia locale dei parametri. Questo conferma  che le strutture utilizzate per effettuare la copia dei parametri sono istanziate all'interno del chip della FPGA, consentendo un esecuzione più efficiente e veloce del kernel.

\vspace{0.3cm}

\clearpage

\section{Execution Summary}
Attraverso l'utilizzo del file \texttt{xrt.ini}, come indicato dalla documentazione \cite{sitoDocumentazioneVitis}, è possibile abilitare la generazione del file \texttt{xrt.run\_summary}. All'interno di questo file sono raccolte le informazioni sugli eventi registrati durante l'esecuzione della FPGA.

\vspace{0.3cm}

\noindent Di seguito l'immagine del \texttt{run\_summary}, aperto tramite lo strumento \texttt{vitis\_analyzer}, delle informazione riguardanti dell'esecuzione del codice spiegato nella sez. \ref{Interprete Kernel} (1 istanza):

\begin{figure}[h!]
\centering
\includegraphics[scale=0.35]{images/Capitolo5/10_im.png}
\caption{run\_summary 1 Kernel}
\label{1curunsummary}
\end{figure}

\noindent All'Interno di questo grafico, possiamo notare come l'esecuzione del programma lato host, che comprende il caricamento del file \texttt{.xclbin}, l'inserimento e l'estrazione dei dati dalla memoria globale, e l'esecuzione effettiva del kernel, dura $413$ ms.
È interessante notare come gran parte del tempo di esecuzione è dedicato alle chiamate API dal lato host (in fig.  \ref{1curunsummary} nella parte superiore), mentre il tempo effettivo dove inizia l'esecuzione della Control Unit (in fig.  \ref{1curunsummary})  evidenziata con nome \texttt{Executions}), avviene in un intervallo molto breve. Ciò è dovuto al fatto che il programma assembly utilizzato per l'esecuzione che possiamo osservare nel grafico in fig. \ref{1curunsummary}, è composto di poche istruzioni e, di fatto, non contiene cicli di queste istruzioni. 

\vspace{0.3cm}

In un ottica futura, nel contesto di una possibile "soft GPU" che sfrutti questi softcore, il caricamento del kernel (contenuto nel file \texttt{.xclbin}) verrebbe eseguito una sola volta per tutti i core, per poi poter caricare volta per volta lo stato della memoria, dei registri, con i relativi programmi da eseguire. Questo permetterebbe di togliere l'overhead associato al dover programmare nuovamente l'FPGA ogni volta che si desidera eseguire un programma.

\chapter{Conclusioni}

L' obbiettivo principale di questa tesi è stato lo sviluppo di un semplice interprete in grado di eseguire un sottoinsieme delle istruzioni del processore Microblaze, compilarlo ed eseguirlo sulla FPGA.  Sono state create differenti versioni del interprete per dimostrare la flessibilità e configurabilità di questa soluzione. Inoltre una delle versioni dell'interprete è stata usata per simulare una GPU con core indipendenti e privi di controllore SIMD.
Il funzionamento di ciascuna versione dell'interprete è stato verificato e validato attraverso test ad-hoc specifici.
Inoltre è stata mostrata e discussa l'occupazione di ogni variante dell'interprete sulla FPGA. Questo ha permesso di stimare il numero massimo di istanze che è possibile inserire all'interno del dispositivo.
In conclusione tutti gli obbiettivi sono stati raggiunti con successo.

\section{Bilancio Personale}
Il lavoro svolto per questa tesi è stato impegnativo e stimolante. L'Affrontare un contesto nuovo come quello delle FPGA, senza particolari conoscenze in materia, ha richiesto un piccolo sforzo nell'acquisire dimestichezza con gli strumenti e le metodologie usate, detto questo la documentazione disponibile grazie alla sua chiarezza e completezza ha costituito un grandissimo supporto durante questo processo.  Inoltre, non è mancata la sfida nel risolvere i problemi che si sono presentati lungo il percorso. Tuttavia, ho potuto sempre contare sul supporto del mio professore. La sua disponibilità e competenza hanno contribuito in maniera significativa al successo di questo progetto. In definitiva, sebbene il lavoro svolto abbia richiesto impegno e e dedizione, l'incontro con il mondo delle FPGA si è rivelato non solo una sfida da affrontare, ma anche un'opportunità unica di crescita personale e accademica. 

 


\appendix

\bibliographystyle{plain}
\bibliography{chapters/Bibliografia.bib}

\chapter{Appendice}
Tutto il Codice seguente è presente in versione integrale nella repo: 

\vspace{0.5cm}

\noindent https://github.com/StangaSimo/tesi

\section{Codice Host 1 Kernel}
\label{codicehost1cu}
\lstinputlisting[language=C++]{listings/code1.cpp} 

\section{Codice Interprete 1 Kernel}
\label{codiceinterprete1cu}
\lstinputlisting[language=C++]{listings/code2.cpp} 

%\section{Codice Host 4 Kernel}
%\label{codicehost1cu}
%\lstinputlisting[language=C++]{listings/code3.cpp} 

\section{Codice Host GPU}
\label{codicehost1cu}
\lstinputlisting[language=C++]{listings/code4.cpp} 

\section{Codice Interprete Floating Point}
\label{codiceinterpretefloating}
\lstinputlisting[language=C++]{listings/code5.cpp} 

\section{File Assembler}
\label{fileassembler}
\begin{itemize}
    \item \texttt{add.s}
    \lstinputlisting{listings/add.s} 
    \item \texttt{branch.s}
    \lstinputlisting{listings/branch.s} 
    \item \texttt{data.s}
    \lstinputlisting{listings/data.s} 
    \item \texttt{sub.s}
    \lstinputlisting{listings/sub.s} 
    \item \texttt{bitop.s}
    \lstinputlisting{listings/bitop.s} 
\end{itemize}

%\section{Istruction from file}
%\label{istructionfromfile}
%\lstinputlisting[language=C]{listings/Capitolo1/code2.c} 




%\noindent \textbf{Ringraziamenti}

\vspace{0.3cm}

\noindent Questa tesi segna la fine di un percorso, nel quale ho imparato e capito tante cose, una molto importante è che da soli non si va da nessuna parte. 
Non posso dire che sia stato tutto merito mio, infatti, il supporto delle persone che mi sono state affianco è stato indispensabile e non si può ripagare, per questo mi limiterò a ringraziarle.

\noindent Ringrazio mia madre e mio padre, senza di voi non ci sarebbe stato niente, se sono la persona di oggi lo devo al vostro aiuto, che non vi siete mai, nemmeno una volta, tirati indietro dal darmi.

\noindent Ringrazio mia sorella e mio fratello, siete sempre stati con me, diventiamo grandi e i cambiamenti si vedono, ma spero di trovarvi per tutta la mia vita al mio fianco.

\noindent Ringrazio Lorenzo, mi hai dato la possibilità di condividere tanto insieme, avventure, persone, ci siamo aiutati a vicenda, sempre, e spero di poter continuare a farlo. 

\noindent Ringrazio Filippo, mi hai dato molto, grazie a questo ti devo la mia crescita in questo periodo, spero di averti ridato anche solo una piccola parte di tutto quello che hai fatto per me, e mi auguro di poter continuare insieme.

\noindent Ringrazio tutte le persone che mi sono state vicino per essere mie amiche, per avermi dato la possibilità di stare con loro, per avermi voluto bene e per avermi aiutato, anche io vi voglio bene, grazie veramente..

\end{document}
% -----------------------------------------------------------------
